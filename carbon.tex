\RequirePackage[l2tabu, orthodox]{nag}
\documentclass[version=3.21, pagesize, twoside=off, bibliography=totoc, DIV=calc, fontsize=12pt, a4paper, french, english]{scrartcl}
\input{preamble/packages}
\input{preamble/redac}
\input{preamble/math_basics}
\input{preamble/math_mine}
%\input{preamble/draw}
%\input{preamble/jdoc}

%I find these settings useful in draft mode. Should be removed for final versions.
	%Which line breaks are chosen: accept worse lines, therefore reducing risk of overfull lines. Default = 200.
		\tolerance=2000
	%Accept overfull hbox up to...
		\hfuzz=2cm
	%Reduces verbosity about the bad line breaks.
		\hbadness 5000
	%Reduces verbosity about the underful vboxes.
		\vbadness=1300

\title{Principles for ecological incitation at the level of a French laboratory \thanks{Thanks.}}
\author{Olivier Cailloux}
\author{Hugo Gilbert}
\affil{Université Paris-Dauphine, PSL Research University, CNRS, LAMSADE, 75016 PARIS, FRANCE\\
	\href{mailto:olivier.cailloux@dauphine.fr}{olivier.cailloux@dauphine.fr}
}
\hypersetup{
	pdfsubject={},
	pdfkeywords={},
}

\begin{document}
\maketitle

\section{Introduction}
\label{sec:intro}
In \cref{sec:alts}, we describe the decision space: the alternatives that we consider. To conceive this we got inspired by initiatives currently used or proposed in various research laboratories, and by the economics literature. In \cref{sec:analysis}, we describe aspects that can be used to decide among alternatives (strengths and weaknesses, circumstances in which some alternatives are infeasible or of low value, …).

\section{Decision space}
\label{sec:alts}
We consider acts, which we consider as an action performed by an actor in some context. We use the term actor to refer to the person performing the action. We consider that a time period is fixed and only consider the acts performed within that time period.

A \emph{pollution tax} 
\footnote{We use the term pollution, for short, to refer to the kind of pollution that generates global warming effects, which includes but does not reduce to carbon emission. This is not to deny that other kinds of pollutions exist (that we do not consider here).} 
can be modeled as a function that associates to each possible act a price in € (hereafter called the tax function); together with a method of using the revenue.

An act can be described using a tuple constituted of the actor characteristics, the action characteristics and the context.
The actor characteristics can be for example the stage of carrier of the actor performing the action.
The action can be described by its usefulness, by its contribution to the normal expected activity of the actor (eg an archeologist may be required to visit some of its areas of specialty), by its pollutant emission… 
Usefulness may mean usefulness for the laboratory, the society, the actor performing the action… It also needs to be specified how this usefulness is determined.
Pollutant emissions may be measured using several dimensions (methane, carbon, other gases) or aggregated using some “carbon equivalent” metric.
The context may include the set of acts performed by that actor previously, within some time period. (This permits to adjust the level of the tax non linearly wrt to repeated actions, for example.) It may also include the total amount of money available to the actor over some period (permitting to set the tax as a proportion of that budget).

Although the tax price of an act may in principle also depend on the acts performed by other actors (the set of acts previously performed by anybody within a given time period would then be part of the context of each act), so as to allow for example tax rates that adjust dynamically depending on a pre-fixed collective target, we do not consider such subtleties here. Rather we consider that the tax function is designed for some time period with some objective in mind and then revised each time period.

The tax function may be independent of one or several of these arguments 
\footnote{A function $f: A × B → C$ depends on $A$ iff $\exists a, a' \in A, b \in B \suchthat f(a, b) ≠ f(a', b)$; otherwise, $f$ is independent of $A$.} 
so this model does not assume that all the listed aspects are to be considered relevant.

A \emph{pollution quota} may be modeled as a function that associates to each possible act a binary decision: allowed or forbidden. Equivalently, it can be modeled as the set of allowed acts. Acts are supposedly described using the same aspects as above.

\subsection{About quotas VS taxes}
A quota function is equivalently modeled using a tax function, assuming that there is some monetary amount that nobody is able to pay, and that we write $\infty$. Such a function uses as codomain only the set $\set{0, \infty}$. Wlog, we thus only talk about tax functions in the rest of this article.

We say that a tax function is non-binary iff its co-domain is different than $\set{0, \infty}$. Note that a tax function is binary iff it corresponds to what is ordinarily called a pollution quota.

\section{What is done elsewhere?} 
Using the bibliographic work of Stephanie Monjon. 

\subsection{Tools to evaluate the carbon footprint}
\begin{itemize}
\item \emph{Universitetet i Bergen}: Implementation of a monitoring of emissions related to mobility with Quantified reduction targets. 
\item \emph{Stockholm Environment Institute}: Staff emissions are being monitored and can be visualized. An online tool to perform these operations is under development.
\end{itemize}

\subsection{Emissions tracking tool and reduction targets}
\begin{itemize}    
\item \emph{ETH Zurich}: Each department defined a target for the reduction of air travel (until 2025, compared to a baseline of 2016-18) and agreed on measures to achieve them. These departments Collaborate with RooteRank to facilitate the comparison of different travel options and the booking of land trips. A travel agency associated with the university is currently trained to choose more environmental options when booking trips.
\item \emph{Zurcher Hochschule der Kunste}: Each department has set its own targets and measures to reduce flight emissions by the end of 2019, exceeding the initial target of a 25\% reduction by the end of 2020 compared to a 2017 baseline.
\end{itemize}

\subsection{Offer advantages when traveling by bus or train}

\begin{itemize}
\item \emph{Universitat Autonoma of Barcelona}: For travels that take less than 10 hours, the employees are encouraged to take the train or the bus. ?These trips are presented by default to them and a travel by plane must be justified. Compensation measures are taken to increase the confort of employees choosing these modes of transport (first class, trips can be extended to add a personal trip to the professional one).
\item \emph{University of Geneve}: Subventions are granted to fund first class tickets for trips by trains that take more than 4 hours. 
\end{itemize}

\subsection{Facilitate / Recommend low-emissive modes of transport}
\begin{itemize}
\item Many universities make it possible to use decision trees to help staff members in choosing their transport mean. 
\item \emph{University of Utrecht}: A decision aiding tool was designed to help in choosing the transportation means chosen by the staff members. ?This tool takes in consideration the traveling time, the carbon footprint associated, and the number of  connections. Night trains and first class options are available. 
\item \emph{EPFL}: The administrative staff received a formation about eco-friendly traveling options. EPFL has also conceived a tool making it possible to compare different traveling options on the criteria: cost, duration, carbon footprint.
\end{itemize}

\subsection{Carbon tax to finance green projects}
\begin{itemize}
\item \emph{Arizona State University}: Since 2018, a taxe of 15\$ for any plane round trip is applied. The money which is collected is used for green compensation projects.
\item \emph{University of Neuchatel}: Since January 2019, a taxe is applied to all trips by planes. The money collected is used to fund activities related to sustainable development as well as projects reducing the carbon footprint of the university.
\item \emph{KU Leuven}: All staff members can, on a voluntary basis, pay a tax of 40 euros by tons of CO2eq for all travels by plane. This money is invested in sustainable activities.
\end{itemize}

\subsection{Quotas on CO2 emissions}
\begin{itemize}
\item \emph{LOCEAN}: Annual individual quotas for the missions organized by the laboratory ($80\%$ of the lab members voted in favor of these quotas in September 2020). These quotas are set to 10 tons of emissions per person in 2021. Quotas are not exchangeable but 4tCO2 can be passed from a year to another. These quotas are meant to decrease with the objective to reduce by half the emissions by 2030. Several exceptions exist: field studies, trips for teaching purposes, and missions of more than 30 days. PhDs and postdocs benefit from an out-of-quota trip every 2 years.
\end{itemize}

\subsection{Forbid flights when alternatives exist}
\begin{itemize}
\item \emph{LOCEAN}: Since automn 2020, it is forbidden to take flights for trips that can be made in less than 5 hours by land means. 
\item \emph{Vrije Universiteit Brussel}: It is forbidden to take flights for trips that can be made in less than 6 hours by land means. Exceptions can be made ?in exceptional circumstances. Avoiding planes is highly recommended for trips which can be made in less than 8 hours otherwise. 
\item \emph{University College London}: Domestic flights are forbidden in the geography department. 
\item \emph{HTW Berlin}: Since 2020, it is forbidden to take flights for trips that can be made in less than 6 hours by land means.
\end{itemize}

 \section{Analysis}
\label{sec:analysis}

Considerations for designing the tax function.

The tax function should be non decreasing with the level of carbon emission, and depend on it. 

A tax has two effects on pollution. One may be named \emph{Reductionism}: it forbids some acts by reducing the available budget of the actor (this effect may occur even if not using $\infty$ prices). The other one may be named \emph{Reactionism}: by using the revenue gathered by the tax, the lab may perform actions that may reduce global warming.

A tax may fail to use Reductionism, by design or unwantedly. This happens iff actors have enough budget to pay for the taxes and still perform the same actions they would have performed without the tax. A tax that fails to use Reductionism is sometimes considered “green washing”, because some believe that Reactionism has no significant impact on reducing the global warming problem.

\subsection{Scope}
An act $a$ is said to be taxed iff $f(a) ≠ 0$ (in other words, the taxed acts is the support of $f$).

Some acts may be considered out of scope (e.g., impossible to tax in practice because the corresponding acts may be hidden easily by the actor, or about which a tax is illegal in some institutional context). Technically, we say that some acts are out of scope iff they are not taxed, and performing them or not never changes anything to the taxes paid about the other acts. Here is the formal definition. Given a set of acts $A$ and an act $a' \notin A$, let $a'_{-A}$ designate the act $a'$ with its context changed so as to not include the acts in $A$. Let $A_i$ designate the set of acts performed by actor $i$ within the considered time period. A set of acts $\mathcal{A}$ is out of scope iff $[\forall a \in \mathcal{A}: f(a) = 0] \land \forall a' \in A_i \setminus \mathcal{A}, \forall A \subseteq \mathcal{A}, f(a'_{-A}) = f(a')$.
An act that is not out of scope is said to be within the scope. Note that an act may be within the scope but not be taxed.

\section{LAMSADE}
\commentOCf{Hugo, je propose d’utiliser cette section comme brouillon pour nos propositions à venir. On pourra éventuellement l’intégrer à l’article à terme en la re-modelant, comme un exemple d’application, mais c’est à voir ultérieurement.}

The tax should use Reductionism.

The scope should include every act which the Dauphine accounting services see (to be defined more precisely?).

The tax should strictly increase with increasing act pollution (implying that no polluting act has zero tax).

Considering that efforts to not-travel are (I think) not linarly proportional to the amount of renouncement (it is easier to give up the first 10\% of ones travels than the last 10\%), it may make sense to tax exponentially (with a reasonably low exponent of course). This may permit, if suitably tuned, to make sure also the rich members of the lab make some effort (to be checked theoretically) and give some of the benefits attributed to tax quotas (at some point the tax is so high that it acts as a practical absolute interdiction). NB some members travel a lot (like Alexis used to), if they have good reasons and we want to not forbid this absolutely then it may be a problem for this proposal.

Allow for exceptions (put some further acts out of scope)? Or reduce the tax according to utility? For example, maybe the director of the lab needs to travel for some official reasons; people (or juniors only) might be encouraged to present their work at conferences; … But this may be taken into account by the increasing tax if sufficiently low at start to allow for occasional travel.

Time period? After some time (years?) the past travels should not count anymore. Or should be discounted.

Differenciate tax on one’s own ANR budget and tax on some lab-or-pole-funded act?

\subsection{Proposition}
Quoi et qui : tout membre du labo pour tout acte passant par les services comptables de Dauphine.

$\gamma(v) = c$, en T eq CO²

Acteur associé à une séquence de voyages $v_i$.

Chaque voyage $v_i$ associé à $c_i = \gamma(v_i)$ et à $t_i$, sa date.

$v_n$ est taxé de : $(\sum_{i = 0}^n c_i \lambda^{t - t_i}) k$, où $k$ est en € / T eq CO² et est fonction de l’ancienneté de l’acteur.

Utilisation de l’argent de la taxe :
- financer mesures incitatives
- Payer 1ère classe
- Compenser différence Avion et moyens de transports moins polluants
- Financer des projets de recherche sur les thématiques environnementales

\subsection{Pour fixer les paramètres de la taxe}
360 k€ (en 2020) : budget moyen projet ANR. Sur trois ans et typiquement trois institutions partenaires comprenant chacune deux chercheurs, ça fait 20 k€ / an / chercheur. Disons budget de 3 k€ / an / chercheur pour les voyages.

Prix typique d’un Paris New York semble être de 400 € A/R.

\subsection{Version exponentielle}
Given $α, β \in \intvl{1, n}$, define $l_α^{t_β} = \sum_{i = 0}^α c_i \lambda^{t_β - t_i}$, 
and let’s write $l_n = l_n^{t_n}$ for short.
Define $l_0 = 0$.

We have $l_n^{t_n} = \sum_{i = 0}^n c_i \lambda^{t_n - t_i} = \sum_{i = 0}^{n - 1} c_i \lambda^{t_n - t_i} + c_n = \frac{\sum_{i = 0}^{n - 1} c_i λ^{t_{n - 1} - t_i}}{λ^{t_{n - 1} - t_n}} + c_n = l_{n - 1}^{t_{n - 1}}λ^{t_n - t_{n - 1}} + c_n$.

The tax of $v_1$ is defined as $\tau(v_1) = \int_0^{c_1} k (1 + \epsilon)^x dx = k \frac{(1 + \epsilon)^{c_1} - 1}{\log(1 + \epsilon)}$.
Letting $α = 1 + ε$, we have $\tau(v_1) = k \frac{α^{c_1} - 1}{\log(α)}$.

The tax of $v_n$ is defined as $\tau(v_n) = \int_{l_{n - 1}}^{l_n} k (1 + \epsilon)^x dx = k (1 + \epsilon)^{l_{n - 1}^{t_n}} \frac{(1 + \epsilon)^{c_n} - 1}{\log(1 + \epsilon)} = k \frac{(1 + \epsilon)^{l_n^{t_n}} - (1 + \epsilon)^{l_{n - 1}^{t_n}} }{\log(1 + \epsilon)}$.

Letting $r = \frac{k}{\log(1 + ε)}$ and $α = 1 + ε$, we get $\tau(v_n) = r (α^{l_{n - 1}^{t_n} + c_n} - α^{l_{n - 1}^{t_n}}) = r (α^{l_n^{t_n}} - α^{l_{n - 1}^{t_n}})$.

Half-life 2 years thus $λ^2 = 1/2$ thus $λ = \frac{\sqrt{2}}{2}$.

\subsection{Loose}
\begin{itemize}
\item What and who should be taxed. Activities that are high emissions should be the ones aimed at. The level of the taxe should be dependent on how much these activities are essential to our jobs. It should be highly dependent on the stage of the career of the person being taxed.
\item What should be the level of the tax, too law is counter-productive, too high would not be accepted.  
\item How should one access the effects of the taxe and at what rate should it be changed; the taxe will probably increase in time; measuring the effects on how many planes are taken is easy.  
\item How to use tax revenues, it should be used to promote greener behaviors, it should not be green washing. 
\end{itemize}

%\bibliography{bibl}

\appendix
\section{Loose}
a tax puts a price on carbon in order to reduce emissions, while a cap puts a limit on the quantity of emissions allowed, thus imputing a price.

Policy design considerations associated with implementing carbon taxes
include determining the tax base, which sectors to tax, where to set the tax rate, how to use tax
revenues, how to assess the impact on consumers and how to ensure that the tax achieves emissions
reduction goals.

critics say that carbon taxes are simply a way of raising government
revenue rather than providing environmental benefits, thus creating a tax rate that would not be economically efficient. For an efficient tax, the rate should be set equal to the marginal damage caused by
carbon emissions.

A carbon tax that rebates revenues can mitigate the impact on low-income households

taxes must be raised when reduction objective is not met (but can be politically difficult). So often prefer C and T.

Because cap and trade is often heralded as a market-based approach in the
political discourse, it is worth noting at the outset that both systems are equally market-based in
the sense that their effectiveness relies in affecting market behavior through emissions pricing.

Ady et al. 08 argue for tax rather than CaT (Cap and Trade). Tax is more economically efficient as it directly equalizes prices accross sources (not applicable for us). Also manages welfare better considering the uncertainty of abatment cost (didn’t get why). Considering redistribution effects: co² taxes are regressive. “Low-income households are more vulnerable to increases in the price of energy-intensive
goods, such as electricity, home heating fuels, and gasoline, because they spend a larger share of
their budget on these items compared with wealthier households.” CaT with free allowances also raise the profits and equity values, which benefits more the higher) income households.

\section{Literature}
For a review of
the advantages and disadvantages of an internationally harmonized carbon tax, many of which are
applicable to domestic carbon taxes, see Nordhaus (2007).

Carbon taxes: a review of experience and policy design considerations JENNY SUMNER, LORI BIRD \& HILLARY DOBOS Pages 922-943 | Published online: 15 Jun 2011 https://doi.org/10.3763/cpol.2010.0093 (Read by OC)

Aldy et al. (Aldy, J.E., Ley, E., Parry, I., 2008, ‘A tax-based approach to slowing global climate change’, National Tax Journal 61(3),
493 – 517, 10.17310/ntj.2008.3.09) discuss the design of carbon taxes
at the domestic level, suggesting that the efficient near-term tax is at least \$5 – 20 per tonne of CO 2 and
that the tax should be placed on upstream sources (Read by OC).
 
Aldy and Pizer (Aldy, J.E., Pizer, W.A., 2009, ‘Issues in designing U.S. climate change policy’, Energy Journal 30(3), 179 – 209.) summarize the litera-
ture on the distributional impacts of carbon taxes, suggesting that recycling revenue to consumers can
reduce the distributional impact.

Metcalf et al. (2008) evaluated three methods of rebating
carbon tax revenue to consumers, and found that a lump sum per capita payment of \$274 made the
carbon tax progressive.

The ultimate design of a carbon tax is subject to the political process, and is thus unlikely to match the theoretically optimal design (see del Río and Labanderia, 2009).

\section{History}
In the early 1990s, carbon taxes emerged
in northern European countries, Finland being the first nation to adopt a carbon tax in 1990. Sub-
sequently, the Netherlands (1990), Norway (1991), Sweden (1991) and Denmark (1992) implemented
carbon taxes. After a lull of almost a decade, the UK began its Climate Change Levy (CCL) in 2001. In
recent years (before 2011), several new taxes have been introduced at the provincial or municipal levels in North
America.
The US House of Representatives
passed H.R.2454 ‘The American Clean Energy and Security Act of 2009’ (the ‘Waxman– Markey Bill’) in
June 2009, and the US Senate introduced S.1733 ‘Clean Energy Jobs and American Power Act’ (the
‘Kerry –Boxer Bill’) in September 2009. While recent federal policy has focused on carbon cap and
trade policies, Congress introduced three carbon tax bills in 2009. 1
the recently proposed French carbon tax was
designed to cover the sectors not addressed by the EU Emissions Trading Scheme (EU ETS).

\section{Petitions} 
\begin{itemize}
\item \emph{Max-Planck-Gesellschaft}: In 2018, an intern petition was launched so that the employees could pledge not to take the plane again for short distances. This petition was signed by 500 persons and was made public. 
\item \emph{Technische Universitat Berlin}: In 2019, the group Scientists4Future of TU Berlin launched a petition asking the teaching community of several universities of Berlin and Brandebourg to not take the plane for short distances. This petition obtained 1700 signatures.
\item \emph{Albert-Ludwigs-Universitat Freiburg}: An open letter was sent to the administration council of the university and signed by employees of the university. This letter was asking for a sustainable travel policy. 
\end{itemize}

\end{document}
