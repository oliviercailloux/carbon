\RequirePackage[l2tabu, orthodox]{nag}
\documentclass[version=3.21, pagesize, twoside=off, bibliography=totoc, DIV=calc, fontsize=12pt, a4paper]{scrartcl}
\input{preamble/packages}
\input{preamble/redac}
\input{preamble/math_basics}
\input{preamble/math_mine}
%\input{preamble/draw}
%\input{preamble/jdoc}

%I find these settings useful in draft mode. Should be removed for final versions.
	%Which line breaks are chosen: accept worse lines, therefore reducing risk of overfull lines. Default = 200.
		\tolerance=2000
	%Accept overfull hbox up to...
		\hfuzz=2cm
	%Reduces verbosity about the bad line breaks.
		\hbadness 5000
	%Reduces verbosity about the underful vboxes.
		\vbadness=1300

\title{Principles for carbon de-incitation at the level of a French laboratory \thanks{Thanks.}}
\author{Olivier Cailloux}
\author{Hugo Gilbert}
\affil{Université Paris-Dauphine, PSL Research University, CNRS, LAMSADE, 75016 PARIS, FRANCE\\
	\href{mailto:olivier.cailloux@dauphine.fr}{olivier.cailloux@dauphine.fr}
}
\hypersetup{
	pdfsubject={},
	pdfkeywords={},
}

\begin{document}
\maketitle

\section{Introduction}
\label{sec:intro}
In \cref{sec:alts}, we describe the decision space: the alternatives that we consider. To conceive this we got inspired by initiatives currently used or proposed in various research laboratories, and by the economics literature. In \cref{sec:analysis}, we describe aspects that can be used to decide among alternatives (strengths and weaknesses, circumstances in which some alternatives are infeasible or of low value, …).

\section{Decision space}
\label{sec:alts}
\begin{itemize}
\item What and who should be taxed. Activities that are high emissions should be the ones aimed at. The level of the taxe should be dependent on how much these activities are essential to our jobs. It should be highly dependent on the stage of the career of the person being taxed.
\item What should be the level of the tax, too law is counter-productive, too high would not be accepted.  
\item How should one access the effects of the taxe and at what rate should it be changed; the taxe will probably increase in time; measuring the effects on how many planes are taken is easy.  
\item How to use tax revenues, it should be used to promote greener behaviors, it should not be green washing. 
\end{itemize}

\section{What is done elsewhere?} 
Using the bibliographic work of Stephanie Monjon. 

\

\textbf{Petitions:} 
\begin{itemize}
\item \emph{Max-Planck-Gesellschaft}: In 2018, an intern petition was launched so that the employees could pledge not to take the plane again for short distances. This petition was signed by 500 persons and was made public. 
\item \emph{Technische Universitat Berlin}: In 2019, the group Scientists4Future of TU Berlin launched a petition asking the teaching community of several universities of Berlin and Brandebourg to not take the plane for short distances. This petition obtained 1700 signatures.
\item \emph{Albert-Ludwigs-Universitat Freiburg}: An open letter was sent to the administration council of the university and signed by employees of the university. This letter was asking for a sustainable travel policy. 
\end{itemize}

\

\textbf{Tools to evaluate the carbon footprint:}
\begin{itemize}
\item \emph{Universitetet i Bergen}: Implementation of a monitoring of emissions related to mobility with Quantified reduction targets. 
\item \emph{Stockholm Environment Institute}: Staff emissions are being monitored and can be visualized. An online tool to perform these operations is under development.
\end{itemize}

\

\textbf{Emissions tracking tool and reduction targets:}
\begin{itemize}    
\item \emph{ETH Zurich}: Each department defined a target for the reduction of air travel (until 2025, compared to a baseline of 2016-18) and agreed on measures to achieve them. These departments Collaborate with RooteRank to facilitate the comparison of different travel options and the booking of land trips. A travel agency associated with the university is currently trained to choose more environmental options when booking trips.
\item \emph{Zurcher Hochschule der Kunste}: Each department has set its own targets and measures to reduce flight emissions by the end of 2019, exceeding the initial target of a 25\% reduction by the end of 2020 compared to a 2017 baseline.
\end{itemize}

\

\textbf{Offer advantages when traveling by bus or train:}

\begin{itemize}
\item \emph{Universitat Autonoma of Barcelona}: For travels that take less than 10 hours, the employees are encouraged to take the train or the bus. ?These trips are presented by default to them and a travel by plane must be justified. Compensation measures are taken to increase the confort of employees choosing these modes of transport (first class, trips can be extended to add a personal trip to the professional one).
\item \emph{University of Geneve}: Subventions are granted to fund first class tickets for trips by trains that take more than 4 hours. 
\end{itemize}

\

\textbf{Facilitate / Recommend low-emissive modes of transport:}
\begin{itemize}
\item Many universities make it possible to use decision trees to help staff members in choosing their transport mean. 
\item \emph{University of Utrecht}: A decision aiding tool was designed to help in choosing the transportation means chosen by the staff members. ?This tool takes in consideration the traveling time, the carbon footprint associated, and the number of  connections. Night trains and first class options are available. 
\item \emph{EPFL}: The administrative staff received a formation about eco-friendly traveling options. EPFL has also conceived a tool making it possible to compare different traveling options on the criteria: cost, duration, carbon footprint.
\end{itemize}

\


 
\section{Analysis}
\label{sec:analysis}

%\bibliography{bibl}

\appendix
\section{Loose}
a tax puts a price on carbon in order to reduce emissions, while a cap puts a limit on the quantity of emissions allowed, thus imputing a price.

Policy design considerations associated with implementing carbon taxes
include determining the tax base, which sectors to tax, where to set the tax rate, how to use tax
revenues, how to assess the impact on consumers and how to ensure that the tax achieves emissions
reduction goals.

critics say that carbon taxes are simply a way of raising government
revenue rather than providing environmental benefits, thus creating a tax rate that would not be economically efficient. For an efficient tax, the rate should be set equal to the marginal damage caused by
carbon emissions.

A carbon tax that rebates revenues can mitigate the impact on low-income households

taxes must be raised when reduction objective is not met (but can be politically difficult). So often prefer C and T.

Because cap and trade is often heralded as a market-based approach in the
political discourse, it is worth noting at the outset that both systems are equally market-based in
the sense that their effectiveness relies in affecting market behavior through emissions pricing.

Ady et al. 08 argue for tax rather than CaT (Cap and Trade). Tax is more economically efficient as it directly equalizes prices accross sources (not applicable for us). Also manages welfare better considering the uncertainty of abatment cost (didn’t get why). Considering redistribution effects: co² taxes are regressive. “Low-income households are more vulnerable to increases in the price of energy-intensive
goods, such as electricity, home heating fuels, and gasoline, because they spend a larger share of
their budget on these items compared with wealthier households.” CaT with free allowances also raise the profits and equity values, which benefits more the higher) income households.

\section{Literature}
For a review of
the advantages and disadvantages of an internationally harmonized carbon tax, many of which are
applicable to domestic carbon taxes, see Nordhaus (2007).

Carbon taxes: a review of experience and policy design considerations JENNY SUMNER, LORI BIRD \& HILLARY DOBOS Pages 922-943 | Published online: 15 Jun 2011 https://doi.org/10.3763/cpol.2010.0093 (Read by OC)

Aldy et al. (Aldy, J.E., Ley, E., Parry, I., 2008, ‘A tax-based approach to slowing global climate change’, National Tax Journal 61(3),
493 – 517, 10.17310/ntj.2008.3.09) discuss the design of carbon taxes
at the domestic level, suggesting that the efficient near-term tax is at least \$5 – 20 per tonne of CO 2 and
that the tax should be placed on upstream sources (Read by OC).
 
Aldy and Pizer (Aldy, J.E., Pizer, W.A., 2009, ‘Issues in designing U.S. climate change policy’, Energy Journal 30(3), 179 – 209.) summarize the litera-
ture on the distributional impacts of carbon taxes, suggesting that recycling revenue to consumers can
reduce the distributional impact.

Metcalf et al. (2008) evaluated three methods of rebating
carbon tax revenue to consumers, and found that a lump sum per capita payment of \$274 made the
carbon tax progressive.

The ultimate design of a carbon tax is subject to the political process, and is thus unlikely to match the theoretically optimal design (see del Río and Labanderia, 2009).

\section{History}
In the early 1990s, carbon taxes emerged
in northern European countries, Finland being the first nation to adopt a carbon tax in 1990. Sub-
sequently, the Netherlands (1990), Norway (1991), Sweden (1991) and Denmark (1992) implemented
carbon taxes. After a lull of almost a decade, the UK began its Climate Change Levy (CCL) in 2001. In
recent years (before 2011), several new taxes have been introduced at the provincial or municipal levels in North
America.
The US House of Representatives
passed H.R.2454 ‘The American Clean Energy and Security Act of 2009’ (the ‘Waxman– Markey Bill’) in
June 2009, and the US Senate introduced S.1733 ‘Clean Energy Jobs and American Power Act’ (the
‘Kerry –Boxer Bill’) in September 2009. While recent federal policy has focused on carbon cap and
trade policies, Congress introduced three carbon tax bills in 2009. 1
the recently proposed French carbon tax was
designed to cover the sectors not addressed by the EU Emissions Trading Scheme (EU ETS).

\end{document}

