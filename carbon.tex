\RequirePackage[l2tabu, orthodox]{nag}
\documentclass[version=3.21, pagesize, twoside=off, bibliography=totoc, DIV=calc, fontsize=12pt, a4paper, french, english]{scrartcl}
%Permits to copy eg x ⪰ y ⇔ v(x) ≥ v(y) from PDF to unicode data, and to search. From pdfTeX users manual. See https://tex.stackexchange.com/posts/comments/1203887.
	\input glyphtounicode
	\pdfgentounicode=1
%Latin Modern has more glyphs than Computer Modern, such as diacritical characters. fntguide commands to load the font before fontenc, to prevent default loading of cmr.
	\usepackage{lmodern}
%Encode resulting accented characters correctly in resulting PDF, permits copy from PDF.
	\usepackage[T1]{fontenc}
%UTF8 seems to be the default in recent TeX installations, but not all, see https://tex.stackexchange.com/a/370280.
	\usepackage[utf8]{inputenc}
%Provides \newunicodechar for easy definition of supplementary UTF8 characters such as → or ≤ for use in source code.
	\usepackage{newunicodechar}
%Text Companion fonts, much used together with CM-like fonts. Provides \texteuro and commands for text mode characters such as \textminus, \textrightarrow, \textlbrackdbl.
	\usepackage{textcomp}
%St Mary’s Road symbol font, used for ⟦ = \llbracket. The \SetSymbolFont command avoids spurious warnings, but also some valid ones, see https://tex.stackexchange.com/a/106719/.
	\usepackage{stmaryrd}\SetSymbolFont{stmry}{bold}{U}{stmry}{m}{n}
%Solves bug in lmodern, https://tex.stackexchange.com/a/261188; probably useful only for unusually big font sizes; and probably better to use exscale instead. Note that the authors of exscale write against this trick.
	%\DeclareFontShape{OMX}{cmex}{m}{n}{
		%<-7.5> cmex7
		%<7.5-8.5> cmex8
		%<8.5-9.5> cmex9
		%<9.5-> cmex10
	%}{}
	%\SetSymbolFont{largesymbols}{normal}{OMX}{cmex}{m}{n}
%More symbols (such as \sum) available in bold version, see https://github.com/latex3/latex2e/issues/71. In article mode (but not in presentation mode), also hides some potentially useful warnings such as when using $\bm{\llbracket}$, see stmaryrd in this document (not sure why).
	\DeclareFontShape{OMX}{cmex}{bx}{n}{%
	   <->sfixed*cmexb10%
	   }{}
	\SetSymbolFont{largesymbols}{bold}{OMX}{cmex}{bx}{n}
%For small caps also in italics, see https://tex.stackexchange.com/questions/32942/italic-shape-needed-in-small-caps-fonts, https://tex.stackexchange.com/questions/284338/italic-small-caps-not-working.
	\usepackage{slantsc}
	\AtBeginDocument{%
		%“Since nearly no font family will contain real italic small caps variants, the best approach is to substitute them by slanted variants.” -- slantsc doc
		%\DeclareFontShape{T1}{lmr}{m}{scit}{<->ssub*lmr/m/scsl}{}%
		%There’s no bold small caps in Latin Modern, we switch to Computer Modern for bold small caps, see https://tex.stackexchange.com/a/22241
		%\DeclareFontShape{T1}{lmr}{bx}{sc}{<->ssub*cmr/bx/sc}{}%
		%\DeclareFontShape{T1}{lmr}{bx}{scit}{<->ssub*cmr/bx/scsl}{}%
	}
%Warn about missing characters.
	\tracinglostchars=2
%Nicer tables: provides \toprule, \midrule, \bottomrule.
	%\usepackage{booktabs}
%For new column type X which stretches; can be used together with booktabs, see https://tex.stackexchange.com/a/97137. “tabularx modifies the widths of the columns, whereas tabular* modifies the widths of the inter-column spaces.” Loads array.
	%\usepackage{tabularx}
%math-mode version of "l" column type. Requires \usepackage{array}.
	%\usepackage{array}
	%\newcolumntype{L}{>{$}l<{$}}
%Provides \xpretocmd and loads etoolbox which provides \apptocmd, \patchcmd, \newtoggle… Also loads xparse, which provides \NewDocumentCommand and similar commands intended as replacement of \newcommand in LaTeX3 for defining commands (see https://tex.stackexchange.com/q/98152 and https://github.com/latex3/latex2e/issues/89).
	\usepackage{xpatch}
%for \nexists and because it is a basic package nowadays, see https://tex.stackexchange.com/q/539592/.
	\usepackage{amssymb}
%loads and fixes some bugs in amsmath (a basic, mandatory package nowadays, see Grätzer, More Math Into LaTeX) and provides \DeclarePairedDelimiter. I recommend \begin{equation}, which allows numbering, rather than \[ (and $$ should be avoided), see https://tex.stackexchange.com/questions/503. Relatedly, do not use the displaymath environment: use equation. Do not use the eqnarray environment: use align. This improves spacing. (See l2tabu or amsldoc.)
	\usepackage{mathtools}
%Package frenchb asks to load natbib before babel-french. Package hyperref asks to load natbib before hyperref.
	\usepackage{natbib}

\newtoggle{LCpres}
	\newtoggle{LCart}
	\newtoggle{LCposter}
	\makeatletter
	\@ifclassloaded{beamer}{
		\toggletrue{LCpres}
		\togglefalse{LCart}
		\togglefalse{LCposter}
		\wlog{Presentation mode}
	}{
		\@ifclassloaded{tikzposter}{
			\toggletrue{LCposter}
			\togglefalse{LCpres}
			\togglefalse{LCart}
			\wlog{Poster mode}
		}{
			\toggletrue{LCart}
			\togglefalse{LCpres}
			\togglefalse{LCposter}
			\wlog{Article mode}
		}
	}
	\makeatother%

%Language options ([french, english]) should be on the document level (last is main); except with tikzposter: put [french, english] options next to \usepackage{babel} to avoid warning. beamer uses the \translate command for the appendix: omitting babel results in a warning, see https://github.com/josephwright/beamer/issues/449. Babel also seems required for \refname.
	\usepackage{babel}
	\frenchbsetup{AutoSpacePunctuation=false}
%https://ctan.org/pkg/amsmath recommends ntheorem, which supersedes amsthm, which corrects the spacing of proclamations and allows for theoremstyle, but I decided to switch to amsthm with thmtools (mentioned in amsthm doc) because ntheorem “seems essentially unmaintaned and has severe problems”, see https://tex.stackexchange.com/q/535950. Must be loaded after amsmath (from amsthm doc).
		\usepackage{amsthm}
		\usepackage{thmtools}
%listings (1.7) does not allow multi-byte encodings. listingsutf8 works around this only for characters that can be represented in a known one-byte encoding and only for \lstinputlisting. Other workarounds: use literate mechanism; or escape to LaTeX (but breaks alignment).
	%\usepackage{listings}
	%\lstset{tabsize=2, basicstyle=\ttfamily, escapechar=§, literate={é}{{\'e}}1}
%I favor acro over acronym because the former is more recently updated (2018 VS 2015 at time of writing); has a longer user manual (about 40 pages VS 6 pages if not counting the example and implementation parts); has a command for capitalization; and acronym suffers a nasty bug when ac used in section, see https://tex.stackexchange.com/q/103483 (though this might be the fault of the silence package and might be solved in more recent versions, I do not know) and from a bug when used with cleveref, see https://tex.stackexchange.com/q/71364. However, loading it makes compilation time (one pass on this template) go from 0.6 to 1.4 seconds, see https://bitbucket.org/cgnieder/acro/issues/115.
	\usepackage{acro}
	%“All options of acro that have not been mentioned in section 4.1 have to be set up… with… \acsetup{…}” -- acro package doc, cited by the Overleaf support (thanks to them!)
	\acsetup{single}
	\DeclareAcronym{AMCD}{short=AMCD, long={Aide Multicritère à la Décision}}
\DeclareAcronym{AHP}{short=AHP, long={Analytic Hierarchy Process}}
\DeclareAcronym{AR}{short=AR, long={Argumentative Recommender}}
\DeclareAcronym{DA}{short=DA, long={Decision Analysis}}
\DeclareAcronym{DJ}{short=DJ, long={Deliberated Judgment}}
\DeclareAcronym{DM}{short=DM, long={Decision Maker}}
\DeclareAcronym{DP}{short=DP, long={Deliberated Preference}}
\DeclareAcronym{MAVT}{short=MAVT, long={Multiple Attribute Value Theory}}
\DeclareAcronym{MCDA}{short=MCDA, long={Multicriteria Decision Aid}}
\DeclareAcronym{MIP}{short=MIP, long={Mixed Integer Program}}
\DeclareAcronym{SCR}{short=SCR, long={Social Choice Rule}}
\DeclareAcronym{SEU}{short=SEU, long={Subjective Expected Utility}}


\iftoggle{LCpres}{
	%I favor fmtcount over nth because it is loaded by datetime anyway; and fmtcount warns about possible conflicts when loaded after nth (“\ordinal already defined use \FCordinal instead”). See also https://english.stackexchange.com/questions/93008.
	\usepackage{fmtcount}
	%For nice input of date of presentation. Must be loaded after the babel package. Has possible problems with srcletter: https://golatex.de/verwendung-von-babel-und-datetime-in-scrlttr2-schlaegt-fehlt-t14779.html.
	\usepackage[nodayofweek]{datetime}
}{
}
%For presentations, Beamer implicitely uses the pdfusetitle option. autonum doc mandates option hypertexnames=false. I want to highlight links only if necessary for the reader to recognize it as a link, to reduce distraction. In presentations, this is already taken care of by beamer (https://tex.stackexchange.com/a/262014). If using colorlinks=true in a presentation, see https://tex.stackexchange.com/q/203056. Crashes the first compilation with tikzposter, just compile again and the problem disappears, see https://tex.stackexchange.com/q/254257.
\makeatletter
\iftoggle{LCpres}{
	\usepackage{hyperref}
}{
	\usepackage[hypertexnames=false, pdfusetitle, linkbordercolor={1 1 1}, citebordercolor={1 1 1}, urlbordercolor={1 1 1}]{hyperref}
	%https://tex.stackexchange.com/a/466235
	\pdfstringdefDisableCommands{%
		\let\thanks\@gobble
	}
}
\makeatother
%urlbordercolor is used both for \url and \doi, which I think shouldn’t be colored, and for \href, thus might want to color manually when required. Requires xcolor.
	\NewDocumentCommand{\hrefblue}{mm}{\textcolor{blue}{\href{#1}{#2}}}
%hyperref doc says: “Package bookmark replaces hyperref’s bookmark organization by a new algorithm (...) Therefore I recommend using this package”.
	\usepackage{bookmark}
%Need to invoke hyperref explicitly to link to line numbers: \hyperlink{lintarget:mylinelabel}{\ref*{lin:mylinelabel}}, with \ref* to disable automatic link. Also see https://tex.stackexchange.com/q/428656 for referencing lines from another document.
	%\usepackage{lineno}
	%\NewDocumentCommand{\llabel}{m}{\hypertarget{lintarget:#1}{}\linelabel{lin:#1}}
	%\setlength\linenumbersep{9mm}
%For complex authors blocks. Seems like authblk wants to be later than hyperref, but sooner than silence. See https://tex.stackexchange.com/q/475513 for the patch to hyperref pdfauthor.
	\ExplSyntaxOn
	\seq_new:N \g_oc_hrauthor_seq
	\NewDocumentCommand{\addhrauthor}{m}{
		\seq_gput_right:Nn \g_oc_hrauthor_seq { #1 }
	}
	\NewExpandableDocumentCommand{\hrauthor}{}{
		\seq_use:Nn \g_oc_hrauthor_seq {,~}
	}
	\ExplSyntaxOff
	{
		\catcode`#=11\relax
		\gdef\fixauthor{\xpretocmd{\author}{\addhrauthor{#2}}{}{}}%
	}
	\iftoggle{LCart}{
		\usepackage{authblk}
		\renewcommand\Affilfont{\small}
		\fixauthor
		\AtBeginDocument{
		    \hypersetup{pdfauthor={\hrauthor}}
		}
	}{
	}
%I do not use floatrow, because it requires an ugly hack for proper functioning with KOMA script (see scrhack doc). Instead, the following command centers all floats (using \centering, as the center environment adds space, http://texblog.net/latex-archive/layout/center-centering/), and I manually place my table captions above and figure captions below their contents (https://tex.stackexchange.com/a/3253).
	\makeatletter
	\g@addto@macro\@floatboxreset\centering
	\makeatother
%Permits to customize enumeration display and references
	%\nottoggle{LCpres}{
		%\usepackage{enumitem} %follow list environments by a string to customize enumeration, example: \begin{description}[itemindent=8em, labelwidth=!] or \begin{enumerate}[label=({\roman*}), ref={\roman*}].
	%}{
	%}
%Provides \Centering, \RaggedLeft, and \RaggedRight and environments Center, FlushLeft, and FlushRight, which allow hyphenation. With tikzposter, seems to cause 1=1 to be printed in the middle of the poster.
	%\usepackage{ragged2e}
%To typeset units by closely following the “official” rules.
	%\usepackage[strict]{siunitx}
%Turns the doi provided by some bibliography styles into URLs.
	\usepackage{doi}
%Makes sure upper case greek letters are italic as well.
	\usepackage{fixmath}
%Provides \mathbb; obsoletes latexsym (see http://tug.ctan.org/macros/latex/base/latexsym.dtx). Relatedly, \usepackage{eucal} to change the mathcal font and \usepackage[mathscr]{eucal} (apparently equivalent to \usepackage[mathscr]{euscript}) to supplement \mathcal with \mathscr. This last option is not very useful as both fonts are similar, and the intent of the authors of eucal was to provide a replacement to mathcal (see doc euscript). Also provides \mathfrak for supplementary letters.
	\usepackage{amsfonts}
%Provides a beautiful (IMHO) \mathscr and really different than \mathcal, for supplementary uppercase letters. But there is no bold version. Alternative: mathrsfs (more slanted), but when used with tikzposter, it warns about size substitution, see https://tex.stackexchange.com/q/495167.
	\usepackage[scr]{rsfso}
%Multiple means to produce bold math: \mathbf, \boldmath (defined to be \mathversion{bold}, see fntguide), \pmb, \boldsymbol (all legacy, from LaTeX base and AMS), \bm (the most recommended one), \mathbold from package fixmath (I don’t see its advantage over \boldsymbol).
%“The \boldsymbol command is obtained preferably by using the bm package, which provides a newer, more powerful version than the one provided by the amsmath package. Generally speaking, it is ill-advised to apply \boldsymbol to more than one symbol at a time.” — AMS Short math guide. “If no bold font appears to be available for a particular symbol, \bm will use ‘poor man’s bold’” — bm. It is “best to load the package after any packages that define new symbol fonts” – bm. bm defines \boldsymbol as synonym to \bm. \boldmath accesses the correct font if it exists; it is used by \bm when appropriate. See https://tex.stackexchange.com/a/10643 and https://github.com/latex3/latex2e/issues/71 for some difficulties with \bm.
	\usepackage{bm}
	\nottoggle{LCpres}{
	%Provides \cref. Unfortunately, cref fails when the language is French and referring to a label whose name contains a colon (https://tex.stackexchange.com/q/83798). Use \cref{sec\string:intro} to work around this. cleveref should go “laster” than hyperref.
		\usepackage{cleveref}
	}{
	}
	\nottoggle{LCposter}{
	%Equations get numbers iff they are referenced. Loading order should be “amsmath → hyperref → cleveref → autonum”, according to autonum doc. Use this in preference to the showonlyrefs option from mathtools, see https://tex.stackexchange.com/q/459918 and autonum doc. See https://tex.stackexchange.com/a/285953 for the etex line. Incompatible with my version of tikzposter (produces “! Improper \prevdepth”). This removes the starred versions, such as equation*. Unfortunately, this prevents using \qedhere in an equation ending a proof, see https://tex.stackexchange.com/q/133358/.
		\expandafter\def\csname ver@etex.sty\endcsname{3000/12/31}\let\globcount\newcount
		\usepackage{autonum}
	}{
	}
%Also loaded by tikz.
	\usepackage{xcolor}
\iftoggle{LCpres}{
	\usepackage{tikz}
	%\usetikzlibrary{babel, matrix, fit, plotmarks, calc, trees, shapes.geometric, positioning, plothandlers, arrows, shapes.multipart}
}{
}
%Vizualization, on top of TikZ
	%\usepackage{pgfplots}
	%\pgfplotsset{compat=1.14}
\usepackage{graphicx}
	\graphicspath{{graphics/}}

%Provides \printlength{length}, useful for debugging.
	%\usepackage{printlen}
	%\uselengthunit{mm}

\iftoggle{LCpres}{
	\usepackage{appendixnumberbeamer}
	%I have yet to see anyone actually use these navigation symbols; let’s disable them
	\setbeamertemplate{navigation symbols}{} 
	\usepackage{preamble/beamerthemeParisFrance}
	\setcounter{tocdepth}{10}
}{
}

\usepackage{amsmath}

%Requires package xcolor.
%\definecolor{ao(english)}{rgb}{0.0, 0.5, 0.0}
\definecolor{chocolate}{rgb}{0.82,0.41,0.12}
\newcommand{\chocolate}[1]{\textcolor{chocolate}{#1}}
%I observed that two braces around \small are required to make it local.
\NewDocumentCommand{\commentOC}{m}{\textcolor{blue}{{\small$\big[$OC: #1$\big]$}}}
%Requires package babel and option [french]. According to babel doc, need two braces around \selectlanguage to make the changes really local.
\NewDocumentCommand{\commentOCf}{m}{\textcolor{blue}{{\small\selectlanguage{french}$\big[$OC : #1$\big]$}}}
\NewDocumentCommand{\commentHG}{m}{\textcolor{red}{{\small$\big[$HG: #1$\big]$}}}
\NewDocumentCommand{\commentHGf}{m}{\textcolor{red}{{\small\selectlanguage{french}$\big[$HG : #1$\big]$}}}

\bibliographystyle{abbrvnat}
\NewDocumentCommand{\possessivecite}{O{}m}{\citeauthor{#2}’s \citeyearpar[#1]{#2}}
\NewDocumentCommand{\Possessivecite}{O{}m}{\Citeauthor{#2}’s \citeyearpar[#1]{#2}}

\iftoggle{LCart}{
	\declaretheorem{theorem}
	\declaretheorem{lemma}
	\declaretheorem{proposition}
	\declaretheorem{conjecture}
	\declaretheorem{corollary}
%Possibly: qed=\scalebox{1.2}{$\hexagon$} (and \usepackage{wasysym}, after amssymb). But probably not useful as most definitions are one § only. Might also consider: circledast, circledcirc, circleddash; black triangle or black circle; diamond…
	\declaretheorem[style=definition]{definition}
	\declaretheorem[style=remark, qed=$\triangle$]{example}
	\declaretheorem[style=remark, qed=$\triangle$]{remark}
}

%https://tex.stackexchange.com/a/467188, https://tex.stackexchange.com/a/36088 - uncomment if one of those symbols is used.
%\DeclareFontFamily{U} {MnSymbolD}{}
%\DeclareFontShape{U}{MnSymbolD}{m}{n}{
%  <-6> MnSymbolD5
%  <6-7> MnSymbolD6
%  <7-8> MnSymbolD7
%  <8-9> MnSymbolD8
%  <9-10> MnSymbolD9
%  <10-12> MnSymbolD10
%  <12-> MnSymbolD12}{}
%\DeclareFontShape{U}{MnSymbolD}{b}{n}{
%  <-6> MnSymbolD-Bold5
%  <6-7> MnSymbolD-Bold6
%  <7-8> MnSymbolD-Bold7
%  <8-9> MnSymbolD-Bold8
%  <9-10> MnSymbolD-Bold9
%  <10-12> MnSymbolD-Bold10
%  <12-> MnSymbolD-Bold12}{}
%\DeclareSymbolFont{MnSyD} {U} {MnSymbolD}{m}{n}
%\DeclareMathSymbol{\ntriplesim}{\mathrel}{MnSyD}{126}
%\DeclareMathSymbol{\nlessgtr}{\mathrel}{MnSyD}{192}
%\DeclareMathSymbol{\ngtrless}{\mathrel}{MnSyD}{193}
%\DeclareMathSymbol{\nlesseqgtr}{\mathrel}{MnSyD}{194}
%\DeclareMathSymbol{\ngtreqless}{\mathrel}{MnSyD}{195}
%\DeclareMathSymbol{\nlesseqgtrslant}{\mathrel}{MnSyD}{198}
%\DeclareMathSymbol{\ngtreqlessslant}{\mathrel}{MnSyD}{199}
%\DeclareMathSymbol{\npreccurlyeq}{\mathrel}{MnSyD}{228}
%\DeclareMathSymbol{\nsucccurlyeq}{\mathrel}{MnSyD}{229}
%\DeclareFontFamily{U} {MnSymbolA}{}
%\DeclareFontShape{U}{MnSymbolA}{m}{n}{
%  <-6> MnSymbolA5
%  <6-7> MnSymbolA6
%  <7-8> MnSymbolA7
%  <8-9> MnSymbolA8
%  <9-10> MnSymbolA9
%  <10-12> MnSymbolA10
%  <12-> MnSymbolA12}{}
%\DeclareFontShape{U}{MnSymbolA}{b}{n}{
%  <-6> MnSymbolA-Bold5
%  <6-7> MnSymbolA-Bold6
%  <7-8> MnSymbolA-Bold7
%  <8-9> MnSymbolA-Bold8
%  <9-10> MnSymbolA-Bold9
%  <10-12> MnSymbolA-Bold10
%  <12-> MnSymbolA-Bold12}{}
%\DeclareSymbolFont{MnSyA} {U} {MnSymbolA}{m}{n}
%%Rightwards wave arrow: ↝. Alternative: \rightsquigarrow from amssymb, but it’s uglier
%\DeclareMathSymbol{\rightlsquigarrow}{\mathrel}{MnSyA}{160}

%03B1 Greek Small Letter Alpha
\newunicodechar{α}{\alpha}
\newunicodechar{β}{\beta}
%03B3 Greek Small Letter Gamma
\newunicodechar{γ}{\gamma}
%03B4 Greek Small Letter Delta
\newunicodechar{δ}{\delta}
\newunicodechar{ε}{\epsilon}
\newunicodechar{λ}{\lambda}
\newunicodechar{μ}{\mu}
\newunicodechar{ν}{\nu}
%03C6 Greek Small Letter Phi (“the ordinary Greek letter, showing considerable glyph variation; in mathematical contexts, the loopy glyph is preferred, to contrast with 03D5 φ” – The Unicode Standard 13.0, https://www.unicode.org/charts/PDF/U0370.pdf)
\newunicodechar{φ}{\phi}
%03D5 Greek Phi symbol (“used as a technical symbol, with a stroked glyph” – The Unicode Standard 13.0, https://www.unicode.org/charts/PDF/U0370.pdf)
\newunicodechar{ϕ}{\varphi}
%03A6 Φ Greek Capital Letter Phi
\newunicodechar{Φ}{\Phi}
%03A8 Φ Greek Capital Letter Psi
\newunicodechar{Ψ}{\Psi}
%2115 Double-Struck Capital N
\newunicodechar{ℕ}{\mathbb{N}}
%211A Double-Struck Capital Q
\newunicodechar{ℚ}{\mathbb{Q}}
%211D Double-Struck Capital R
\newunicodechar{ℝ}{\mathbb{R}}
%2194 
\newunicodechar{↔}{\leftrightarrow}
%21CF Rightwards Double Arrow with Stroke
\newunicodechar{⇏}{\nRightarrow}
%21D0 Leftwards Double Arrow
\newunicodechar{⇐}{\ensuremath{\Leftarrow}}
%21D2 Rightwards Double Arrow
\newunicodechar{⇒}{\ensuremath{\Rightarrow}}
%21D4 Left Right Double Arrow
\newunicodechar{⇔}{\Leftrightarrow}
%21DD Rightwards Squiggle Arrow
\newunicodechar{⇝}{\rightsquigarrow}
%2205 Empty Set
\newunicodechar{∅}{\emptyset}
%2212 Minus Sign
\newunicodechar{−}{\ifmmode{-}\else\textminus\fi}
%221E Infinity
\newunicodechar{∞}{\infty}
%2227 Logical And
\newunicodechar{∧}{\land}
%2228 Logical Or
\newunicodechar{∨}{\lor}
%2229 Intersection
\newunicodechar{∩}{\cap}
%222A Union
\newunicodechar{∪}{\cup}
%2260 Not Equal To (handy also as text in informal writing)
\newunicodechar{≠}{\ensuremath{\neq}}
%2264 Less-Than or Equal To
\newunicodechar{≤}{\leq}
%2265 Greater-Than or Equal To
\newunicodechar{≥}{\geq}
%2270 Neither Less-Than nor Equal To
\newunicodechar{≰}{\nleq}
%2271 Neither Greater-Than nor Equal To
\newunicodechar{≱}{\ngeq}
%2272 Less-Than or Equivalent To
\newunicodechar{≲}{\lesssim}
%2273 Greater-Than or Equivalent To
\newunicodechar{≳}{\gtrsim}
%2274 Neither Less-Than nor Equivalent To – also, from MnSymbol: \nprecsim, a more exact match to the Unicode symbol; and \npreccurlyeq, too small
\newunicodechar{≴}{\not\preccurlyeq}
%2275 Neither Greater-Than nor Equivalent To
\newunicodechar{≵}{\not\succcurlyeq}
%2279 Neither Greater-Than nor Less-Than – requires MnSymbol; also \nlessgtr from txfonts/pxfonts, \ngtreqless from MnSymbol (but much higher), \ngtrless from MnSymbol (a more exact match to the Unicode symbol); for incomparability (not matching this Unicode symbol), may also consider \ntriplesim from MnSymbol,\nparallelslant from fourier, \between from mathabx, or ⋈
\newunicodechar{≹}{\ngtreqlessslant}
%227A Precedes
\newunicodechar{≺}{\prec}
%227B Succeeds
\newunicodechar{≻}{\succ}
%227C Precedes or Equal To
\newunicodechar{≼}{\preccurlyeq}
%227D Succeeds or Equal To
\newunicodechar{≽}{\succcurlyeq}
%227E Precedes or Equivalent To
\newunicodechar{≾}{\precsim}
%227F Succeeds or Equivalent To
\newunicodechar{≿}{\succsim}
%2280 Does Not Precede
\newunicodechar{⊀}{\nprec}
%2281 Does Not Succeed
\newunicodechar{⊁}{\nsucc}
%2286
\newunicodechar{⊆}{\subseteq}
%22A2
\newunicodechar{⊢}{\vdash}
%22A5 Up Tack
\newunicodechar{⊥}{\bot}
%22AC
\newunicodechar{⊬}{\nvdash}
%22B2 Normal Subgroup Of – using \vartriangleleft from amsfonts, which goes well with \trianglelefteq, \ntriangleright, and so on, also from amsfonts; another possibility is \lhd from latexsym, which seems visually equivalent to \vartriangleleft from amsfonts; latexsym also has ⊴=\unlhd, but doesn’t have some other related symbols. Other related symbols: \triangleleft from latesym package is too small; fdsymbol provides \triangleleft=\medtriangleleft and \vartriangleleft=\smalltriangleleft; MnSymbol provides \medtriangleleft and \vartriangleleft=\lessclosed=\lhd which are smaller than \vartriangleleft from amsfont; \vartriangleleft from mathabx (p. 67), looks different (wider); also \vartriangleleft from boisik (p. 69) looks still different; \vartriangleleft=\lhd from stix are smaller. Oddly enough, \triangleright appears as the LMMathItalic12-Regular font whereas \rhd appears as LASY10 and \vartriangleright appears as MSAM10 in the resulting PDF.
\newunicodechar{⊲}{\vartriangleleft}
%22B3 Contains as Normal Subgroup (also: 25B7 White right-pointing triangle or 25B9 White right-pointing small triangle)
\newunicodechar{⊳}{\vartriangleright}
%22B4 Normal Subgroup of or Equal To
\newunicodechar{⊴}{\trianglelefteq}
%22B5 Contains as Normal Subgroup or Equal To
\newunicodechar{⊵}{\trianglerighteq}
%22C8 Bowtie
\newunicodechar{⋈}{\bowtie}
%22EA Not Normal Subgroup Of
\newunicodechar{⋪}{\ntriangleleft}
%22EB Does Not Contain As Normal Subgroup
\newunicodechar{⋫}{\ntriangleright}
%22EC Not Normal Subgroup of or Equal To
\newunicodechar{⋬}{\ntrianglelefteq}
%22ED Does Not Contain as Normal Subgroup or Equal
\newunicodechar{⋭}{\ntrianglerighteq}
%25A1 White Square
\newunicodechar{□}{\Box}
%2713 Check Mark – amssymb \checkmark; bbding \Checkmark \CheckmarkBold; pifont \ding{51} \ding{52}; dingbat \checkmark; thanks to https://tex.stackexchange.com/a/132785 and https://tex.stackexchange.com/questions/132783#comment299869_132789
\newunicodechar{✓}{\checkmark}
%27E6 Mathematical Left White Square Bracket – requires stmaryrd (alternative: \text{\textlbrackdbl}, but ugly if used in an italicized text such as a theorem)
\newunicodechar{⟦}{\llbracket}
%27E7 Mathematical Right White Square Bracket
\newunicodechar{⟧}{\rrbracket}
%27FC Long Rightwards Arrow from Bar
\newunicodechar{⟼}{\longmapsto}
%2AB0 Succeeds Above Single-Line Equals Sign
\newunicodechar{⪰}{\succeq}
%301A Left White Square Bracket
\newunicodechar{〚}{\textlbrackdbl}
%301B Right White Square Bracket
\newunicodechar{〛}{\textrbrackdbl}
%00B1 Plus-Minus Sign ±
\DeclareUnicodeCharacter{00B1}{\pm}
%← Leftwards Arrow
\DeclareUnicodeCharacter{2190}{\ifmmode\leftarrow\else\textleftarrow\fi}
%→ is defined by default as \textrightarrow, which is invalid in math mode. Same thing for the three other commands. Using \DeclareUnicodeCharacter instead of \newunicodechar because the latter warns about the previous definition.
%→ Rightwards Arrow
\DeclareUnicodeCharacter{2192}{\ifmmode\rightarrow\else\textrightarrow\fi}
%→ Rightwards Double Arrow
\DeclareUnicodeCharacter{21D2}{\Rightarrow}
%¬ Not Sign
\DeclareUnicodeCharacter{00AC}{\ifmmode\lnot\else\textlnot\fi}
%… Horizontal Ellipsis
\DeclareUnicodeCharacter{2026}{\ifmmode\dots\else\textellipsis\fi}
%× Multiplication Sign
\DeclareUnicodeCharacter{00D7}{\ifmmode\times\else\texttimes\fi}
%Permits to really obtain a straight quote when typing a straight quote; potentially dangerous, see https://tex.stackexchange.com/a/521999
\catcode`\'=\active
\DeclareUnicodeCharacter{0027}{\ifmmode^\prime\else\textquotesingle\fi}


\NewDocumentCommand{\N}{}{ℕ}
\NewDocumentCommand{\Q}{}{ℚ}
\NewDocumentCommand{\R}{}{ℝ}
%\mathscr is rounder than \mathcal.
\NewDocumentCommand{\powerset}{m}{\mathscr{P}(#1)}
%Powerset without zero.
\NewDocumentCommand{\powersetz}{m}{\mathscr{P}^*(#1)}
%https://tex.stackexchange.com/a/45732, works within both \set and \set*, same spacing than \mid (https://tex.stackexchange.com/a/52905).
\NewDocumentCommand{\suchthat}{}{\;\ifnum\currentgrouptype=16 \middle\fi|\;}
\NewDocumentCommand{\st}{}{\text{ s.\ t.\ }}
\NewDocumentCommand{\knowing}{}{\;\ifnum\currentgrouptype=16 \middle\fi|\;}
%Integer interval.
\NewDocumentCommand{\intvl}{m}{⟦#1⟧}
%Allows for \abs and \abs*, which resizes the delimiters.
\DeclarePairedDelimiter\abs{\lvert}{\rvert}
\NewDocumentCommand\card{m}{\##1}
%\DeclarePairedDelimiter\card{\lvert}{\rvert}
\DeclarePairedDelimiter\floor{\lfloor}{\rfloor}
\DeclarePairedDelimiter\ceil{\lceil}{\rceil}
%Perhaps should use U+2016 ‖ DOUBLE VERTICAL LINE here?
\DeclarePairedDelimiter\norm{\lVert}{\rVert}
%From mathtools. Better than using the package braket because braket introduces possibly undesirable space. Then: \begin{equation}\set*{x \in \R^2 \suchthat \norm{x}<5}\end{equation}.
\DeclarePairedDelimiter\set{\{}{\}}
\DeclareMathOperator*{\argmax}{arg\,max}
\DeclareMathOperator*{\argmin}{arg\,min}

%UTR #25: Unicode support for mathematics recommend to use the straight form of phi (by default, given by \phi) rather than the curly one (by default, given by \varphi), and thus use \phi for the mathematical symbol and not \varphi. I however prefer the curly form because the straight form is too easy to mix up with the symbol for empty set.
\let\phi\varphi

%The amssymb solution.
\NewDocumentCommand{\restr}{mm}{{#1}_{\restriction #2}}
%Another acceptable solution.
%\NewDocumentCommand{\restr}{mm}{{#1|}_{#2}}
%https://tex.stackexchange.com/a/278631; drawback being that sometimes the text collides with the line below.
%\NewDocumentCommand\restr{mm}{#1\raisebox{-.5ex}{$|$}_{#2}}


\NewDocumentCommand{\allbundles}{}{\mathscr{B}}
\NewDocumentCommand{\allacts}{}{\mathscr{A}}

%Decision Theory (MCDA and SC)
\NewDocumentCommand{\allalts}{}{\mathscr{A}}
\NewDocumentCommand{\allcrits}{}{\mathscr{C}}
\NewDocumentCommand{\alts}{}{A}
\NewDocumentCommand{\dm}{}{i}
\NewDocumentCommand{\allF}{}{\mathscr{F}}
\NewDocumentCommand{\allvoters}{}{\mathscr{N}}
\NewDocumentCommand{\voters}{}{N}
\NewDocumentCommand{\prof}{}{\boldsymbol{P}}
\NewDocumentCommand{\linors}{}{\mathscr{L}(\allalts)}
%Thanks to https://tex.stackexchange.com/q/154549
	%\makeatletter
	%\def\@myRgood@#1#2{\mathrel{R^X_{#2}}}
	%\def\myRgood{\@ifnextchar_{\@myRgood@}{\mathrel{R^X}}}
	%\makeatother
\NewDocumentCommand{\pref}{}{\succ}
\NewDocumentCommand{\prefi}{O{i}}{\succ_{#1}}

%Deliberated Judgment
\NewDocumentCommand{\allargs}{}{S^*}
\NewDocumentCommand{\args}{}{S}
\NewDocumentCommand{\ar}{}{s}
\NewDocumentCommand{\allprops}{}{T}
\NewDocumentCommand{\prop}{}{t}
\NewDocumentCommand{\ileadsto}{}{⇝}
\NewDocumentCommand{\ibeatse}{}{⊳_\exists}
\NewDocumentCommand{\nibeatse}{}{⋫_\exists}
\NewDocumentCommand{\ibeatsst}{}{⊳_\forall}
\NewDocumentCommand{\nibeatsst}{}{⋫_\forall}
\NewDocumentCommand{\mleadsto}{O{\eta}}{⇝_{#1}}
\NewDocumentCommand{\mbeats}{O{\eta}}{⊳_{#1}}
\NewDocumentCommand{\ibeatseinv}{}{⊳_\exists^{-1}}

%Logic
\NewDocumentCommand{\ltru}{}{\texttt{T}}
\NewDocumentCommand{\lfal}{}{\texttt{F}}





\newtheorem{property}{Property}
%\newtheorem{caorrolary}{Corrolary}
%\NewDocumentCommand{\tikzmark}{m}{%
	\tikz[overlay, remember picture, baseline=(#1.base)] \node (#1) {};%
}

\newlength{\GraphsDNodeSep}
\setlength{\GraphsDNodeSep}{7mm}
\tikzset{/GraphsD/dot/.style={
	shape=circle, fill=black, inner sep=0, minimum size=1mm
}}

% MCDA Drawing Sorting
\newlength{\MCDSCatHeight}
\setlength{\MCDSCatHeight}{6mm}
\newlength{\MCDSAltHeight}
\setlength{\MCDSAltHeight}{4mm}
%separation between two vertical alts
\newlength{\MCDSAltSep}
\setlength{\MCDSAltSep}{2mm}
\newlength{\MCDSCatWidth}
\setlength{\MCDSCatWidth}{3cm}
\newlength{\MCDSAltWidth}
\setlength{\MCDSAltWidth}{2.5cm}
\newlength{\MCDSEvalRowHeight}
\setlength{\MCDSEvalRowHeight}{6mm}
\newlength{\MCDSAltsToCatsSep}
\setlength{\MCDSAltsToCatsSep}{1.5cm}
\newcounter{MCDSNbAlts}
\newcounter{MCDSNbCats}
\newlength{\MCDSArrowDownOffset}
\setlength{\MCDSArrowDownOffset}{0mm}
\tikzset{/MCD/S/alt/.style={
	shape=rectangle, draw=black, inner sep=0, minimum height=\MCDSAltHeight, minimum width=\MCDSAltWidth
}}
\tikzset{/MCD/S/pref/.style={
	shape=ellipse, draw=gray, thick
}}
\tikzset{/MCD/S/cat/.style={
	shape=rectangle, draw=black, inner sep=0, minimum height=\MCDSCatHeight, minimum width=\MCDSCatWidth
}}
\tikzset{/MCD/S/evals matrix/.style={
	matrix, row sep=-\pgflinewidth, column sep=-\pgflinewidth, nodes={shape=rectangle, draw=black, inner sep=0mm, text depth=0.5ex, text height=1em, minimum height=\MCDSEvalRowHeight, minimum width=12mm}, nodes in empty cells, matrix of nodes, inner sep=0mm, outer sep=0mm, row 1/.style={nodes={draw=none, minimum height=0em, text height=, inner ysep=1mm}}
}}

%Git
\newlength{\GitDCommitSep}
\setlength{\GitDCommitSep}{13mm}
\tikzset{/GitD/commit/.style={
	shape=rectangle, draw, minimum width=4em, minimum height=0.6cm
}}
\tikzset{/GitD/branch/.style={
	shape=ellipse, draw, red
}}
\tikzset{/GitD/head/.style={
	shape=ellipse, draw, fill=yellow
}}

%Social Choice
\tikzset{/SCD/profile matrix/.style={
	matrix of math nodes, column sep=3mm, row sep=2mm, nodes={inner sep=0.5mm, anchor=base}
}}
\tikzset{/SCD/rank-profile matrix/.style={
	matrix of math nodes, column sep=3mm, row sep=2mm, nodes={anchor=base}, column 1/.style={nodes={inner sep=0.5mm}}, row 1/.style={nodes={inner sep=0.5mm}}
}}
\tikzset{/SCD/rank-vector/.style={
	draw, rectangle, inner sep=0, outer sep=1mm
}}
\tikzset{/SCD/isolated rank-vector/.style={
	draw, matrix of math nodes, column sep=3mm, inner sep=0, matrix anchor=base, nodes={anchor=base, inner sep=.33em}, ampersand replacement=\&
}}

% GUI
\tikzset{/GUID/button/.style={
	rectangle, very thick, rounded corners, draw=black, fill=black!40%, top color=black!70, bottom color=white
}}

% Logger objects
\tikzset{/loggerD/main/.style={
	shape=rectangle, draw=black, inner sep=1ex, minimum height=7mm
}}
\tikzset{/loggerD/helper/.style={
	shape=rectangle, draw=black, dashed, minimum height=7mm
}}
\tikzset{/loggerD/helper line/.style={
	<->, draw, dotted
}}

% Beliefs
\tikzset{/BeliefsD/attacker/.style={
	shape=rectangle, draw, minimum size=8mm
}}
\tikzset{/BeliefsD/supporter/.style={
	shape=circle, draw
}}


%\input{preamble/jdoc}

%I find these settings useful in draft mode. Should be removed for final versions.
	%Which line breaks are chosen: accept worse lines, therefore reducing risk of overfull lines. Default = 200.
		\tolerance=2000
	%Accept overfull hbox up to...
		\hfuzz=2cm
	%Reduces verbosity about the bad line breaks.
		\hbadness 5000
	%Reduces verbosity about the underful vboxes.
		\vbadness=1300

\title{Principles for ecological incitation at the level of a French laboratory \thanks{Thanks.}}
\author{Olivier Cailloux}
\author{Hugo Gilbert}
\affil{Université Paris-Dauphine, PSL Research University, CNRS, LAMSADE, 75016 PARIS, FRANCE\\
	\href{mailto:olivier.cailloux@dauphine.fr}{olivier.cailloux@dauphine.fr}
}
\hypersetup{
	pdfsubject={},
	pdfkeywords={},
}

\begin{document}
\maketitle

\section{Introduction}
\label{sec:intro}
In \cref{sec:alts}, we describe the decision space: the alternatives that we consider. To conceive this we got inspired by initiatives currently used or proposed in various research laboratories, and by the economics literature. In \cref{sec:analysis}, we describe aspects that can be used to decide among alternatives (strengths and weaknesses, circumstances in which some alternatives are infeasible or of low value, …).

\section{Decision space}
\label{sec:alts}
We consider acts, which we consider as an action performed by an actor in some context. 
Let $\allacts$ denote the set of acts.
We use the term actor to refer to the person performing the action. We consider that a time period is fixed and only consider the acts performed within that time period.

A \emph{pollution tax} 
\footnote{We use the term pollution, for short, to refer to the kind of pollution that generates global warming effects, which includes but does not reduce to carbon emission.} 
can be modeled as a function that associates to each possible act a price in € (hereafter called the tax function); together with a method of using the revenue.

\subsection{Indicators}
\label{sec:indic}
An indicator describes an act on some aspect on which the tax associated to the given act may depend.

Indicators are about the actor characteristics, the action characteristics and the context.
\begin{itemize}
	\item Actor characteristics: the stage of carrier of the actor.
	\item The total amount of money available to the actor over some period (permitting to set the tax as a proportion of that budget).
	\item Action characteristics: its usefulness. Usefulness may mean usefulness for the laboratory, the society, the actor performing the action… It also needs to be specified how this usefulness is determined.
	\item Its pollutant emission. Pollutant emissions may be measured using several dimensions (methane, carbon, other gases) or aggregated using some “carbon equivalent” metric.
	\item Context: the set of actions performed by that actor previously, within some time period. (This permits to adjust the level of the tax non linearly wrt to repeated actions, for example.)
	\item The contribution of the action to the normal expected activity of the actor. For example, an archeologist may be required to visit some of its areas of specialty.
\end{itemize}
The tax function may be independent of one or several of these arguments%
\footnote{A function $f: A × B → C$ depends on $A$ iff $\exists a, a' \in A, b \in B \suchthat f(a, b) ≠ f(a', b)$; otherwise, $f$ is independent of $A$.} 
so this model does not require considering all the listed aspects as relevant.

Although the tax price of an act may in principle also depend on the acts performed by other actors (the set of actions performed by anybody within a given time period would then be part of the context of each act), so as to allow for example tax rates that adjust dynamically depending on a pre-fixed collective target, we do not consider such subtleties here. Rather we consider that the tax function is designed for some time period with some objective in mind and then revised each time period.

\section{Analysis}
\label{sec:analysis}
Considerations for designing the tax function.

A tax has two effects on pollution. One may be named \emph{Reductionism}: it forbids some acts by reducing the available budget of the actor (this effect may occur even if not using $\infty$ prices). The other one may be named \emph{Reactionism}: by using the revenue gathered by the tax, the lab may perform actions that may reduce global warming.

A tax may fail to use Reductionism, by design or unwantedly. This happens iff actors have enough budget to pay for the taxes and still perform the same actions they would have performed without the tax. A tax that fails to use Reductionism is sometimes considered “green washing”, because some believe that Reactionism has no significant impact on reducing the global warming effect.

Taxes may spread efforts in a more or less egalitarian manner.

\subsection{Welfare}
Given an integer $i \in \mathbb{N}$, we use $[n]$ to denote the set $\{1,\ldots, n\}$ (thus $[0] = \emptyset$). 

We denote by $\allacts = \allOperations \times \allMembers$ the set of \emph{acts}, where an act $a$ is a tuple $a = (o,m)$, where $o$  is the operation performed, and $m$ is the lab member performing it; $\allOperations$ denotes the set of all possible operations and $\allMembers$ denotes the set of all lab members. By abuse of notation, we denote by $o(a)$ and $m(a)$ the first and second components of act $a$. \HG{Probablement, quand les fonctions dépendent du membre (coût, taxe) elles ne devraient dépendre que du rang de la personne.}

We call \emph{history} a finite sequence of non-repeated acts, denoted generically as $H$. 
The set of all histories $\allHistories$ is the set of finite sequences of acts (including the empty sequence $()$). 
Let $G \in \allHistories$ denote a history of a single member (thus $\forall a, a' \in G: m(a) = m(a')$) and $\allMH$ the set of such single-member-histories.

We define some notations on histories. 
Given a history $H = (a_1,\ldots,a_n)$ and an act $a\not\in H$, we use $H\circ a$ to  denote the history $(a_1,\ldots, a_n, a)$. 
Given a history $H = (a_1,\ldots,a_n)$ and a set $S\subseteq [n]$, we use $H_S$ to denote the sequence that is constructed by using the acts of $H$ of indices in $S$; by abuse of notation we use $H_i$ in places of $H_{\{i\}}$. 
Given a history $H$ and a lab member $m$, we denote by $H^{m}$, the sub-sequence of $H$ compounded of all acts $a\in H$ such that $m(a) = m$. 
Given histories $H$ and $H'$, we write $H'\subseteq H$ iff $H'$ is a subsequence of $H$.
Given $U \subseteq \allacts$,  we use $H - U$ to denote the sequence that is constructed by removing the elements in $U$ from history $H$ without changing the order of the remaining elements.

The lab would like to choose the best history of acts to realize among a set $T \subseteq \allHistories$ which denote the set of possible histories for the current year. 
The preference of the lab is described by a \emph{welfare ordering} $\succeq$ (a linear order) over $\allHistories$.
However, recall that each act $a$ has a bare cost $c(a)$ (in €) and its action $o(a)$ has a pollution level $p(o(a)) = p(a)$ (in CO²eq). 
The choices of the lab are restricted by a budget $M$ in € and a carbon budget $C$ in CO²eq.

A history $H$ is \emph{raw-cost-feasible} iff $\sum_{a\in H} c(a) ≤ M$; it is \emph{CO²-feasible} iff $\sum_{a\in H} p(a) ≤ C$.
Let $F \subseteq T$ denote the \emph{a priori feasible} histories, meaning, the ones of $T$ which are raw-cost-feasible and CO²-feasible.

The lab would like to design a tax function to maximize welfare under the budget constraints. 
Let $τ: \allMH → \R^+$ denote the tax function. (The set $\R^+$ includes zero.)  
%Given a history $H$, the tax function only depends only on the subsequence $H^{m}$ where $m = m(H_{|H|})$ is the lab member making the last operation in the history, $\tau(H) = \tau(H^{m})$. 
Abusing notation, given $H \in \allHistories$, we also write $τ(H)$ to mean $\tau(H^{m})$ where $m = m(H_{|H|})$ is the lab member making the last operation in the history. 
A cap is a value $M' ≥ M$.
%A history $H$ is tax-feasible (for a given cap $M'$) iff $\sum_{a \in H} (c_a) + τ(H) ≤ M'$.
A history $H$ is tax-feasible (for a given cap $M'$) iff given $H = \set{a_1, …, a_k}$, $\sum_{i \in \intvl{1, k}} [(c_{a_i}) + τ(H_{\intvl{1, i}})] ≤ M'$.
\OCf{On taxe tous les voyages, pas seulement le dernier, non ?}

\subsection{Utility per act}
We suppose that the lab associates a utility $u(H)$ to every history 
\HG{is $u$, linear, submodular, should we consider the utility of the sequence or the increment in utility.} 
such that $u(()) = 0$. 
The lab decides to grant the act $a$ given history $H$ according to some function $f(u(H \circ a) - u(H), c(a) + τ(H\circ a)) \in \set{0, 1}$. An act $a$ is granted given history $H$ iff $f(u(H \circ a) - u(H), c(a) + τ( H \circ a)) = 1$. A history $H$ is granted if $H_i$ is granted given $H_{[i-1]}$ for all $i$ in $[|H|]$.

An act $a$ given $H$ \emph{weakly Pareto-dominates} another act $a'$ given $H'$ iff $|H^{m(a)}| = |{H'}^{m(a')}|$ 
and $\forall i: p(H^{m(a)}_i) ≤ p({H'}^{m(a')}_i) \land u(H \circ a) - u(H) ≥ u(H'\circ a') - u(H') \land p(a) ≤ p(a') \land c(a) ≤ c(a')$.

$τ$ is weakly Pareto iff $\tau(H \circ a) \le \tau(H' \circ a')$ when $a$ given $H$ weakly pareto-dominates $a'$ given $H'$.
$f$ is weakly Pareto iff it does not decrease on its first argument and does not increase on its second argument.

Define $g(u(H \circ a) - u(H), c(a)) \in \set{0, 1}$ as a function of $u$ and $c$ (not of $τ$). 
$g$ is weakly Pareto iff it does not decrease on its first argument and does not increase on its second argument.

We wonder in which conditions we obtain the following property. 
$\forall {\succeq}, T, M, C \suchthat F ≠ \emptyset \land \exists g$ that selects a $\succeq$-maximal element of the raw-cost-feasible histories of $T$ with $g$ WP: (P1), that is, $\exists τ, M', f$ such that the maximal sequences of granted acts constitute sequences that constitute a non-empty prefix of $\succeq_{[F]}$ and are tax-feasible. 
We may also require that there is only one maximal sequences of granted acts.

This is not guaranteed.

\begin{example}
	$T = F = \set{H_1 = (a_1, a_2), H_2 = (a)}$ with $m(a_1)\neq m(a_2)\neq m(a)$, 
	$a$ pareto-dominates $a_1$ given history $()$ and $a_2$ given $(a_1)$.
	Utilities are set such that $u(H\circ b) - u(H) = 1$, $\forall H$ and $b \in \{a_1,a _2\}$; $u(H\circ a) - u(H) = 0.5$, $\forall H$.
	$p(a_1) = p(a_2) = 2T$; $p(a) = 3T$; $C = 3T$; and raw costs of all acts are equal, as a result $F = \{H_2\}$.
	$H_1 \succ H_2$.
	Any $g$ WP will pick $H_1$.
	We want to select $H_2$ (as $H_1$ is not CO²-feasible).
	This requires a higher tax on $a_1$ given $()$ and $a_2$ given $(a_1)$ than on $a$ given $()$, which violates WP.
\end{example}

\subsubsection{Without the condition on g}
Here we see that the condition on $g$ is not vacuous.
We wonder in which conditions we obtain the following property.
$\forall {\succeq}, T, M, C \suchthat F ≠ \emptyset: (P1)$

I believe that with no constraint on $τ$ or $f$, the answer is: always.
Consider $B \in \max_F {\succeq}$.
Define $M' = M$ and $\tau(H) = 0$ if $H\subseteq B$ and $\infty$ otherwise. 

In which conditions are $τ$ and $f$ weakly Pareto (WP)?

\begin{example}
	$T = F = \set{H_1 = (a_1, a_2), H_2 = (a)}$ with $m(a_1)\neq m(a_2)\neq m(a)$, 
	$a$ pareto-dominates $a_1$ given history $()$ and $a_2$ given $(a_1)$.
	Utilities are set such that $u(H\circ b) - u(H) = 0.8$, $\forall H$ and $b \in \{a_1,a _2\}$; $u(H\circ a) - u(H) = 1$, $\forall H$.
	$H_1 \succ H_2$,
	pollution and raw costs of all acts are equal.

	We want to select $H_1$.
	Need to put a strictly higher tax on $a$ otherwise by $f$ being weakly pareto, $a$ is taken.
	Thus there is no weakly Pareto $(τ, f)$.
\end{example}
Note that in this example, there is also no $g$ that selects the right history, so it’s not “the tax’ fault”.
A more meaningful question would ensure that $u$ and $g$ appropriately maximise $\succeq$.

\subsubsection{Sub-sequence closed}
Assume that $T$ is \emph{sub-sequence-closed}, meaning that $H \in T ⇒ \forall h \in H: H - \{h\} \in T$.

Does it hold that $\forall {\succeq}, T \text{sub-sequence-closed}, M, C \suchthat F ≠ \emptyset \land \exists g$ that selects a $\succeq$-maximal element of the raw-cost-feasible bundles of $T$ with $g$ WP: (P1). %$\exists τ, M', f$ such that $τ, f$ are WP and the set of granted acts constitute a bundle that is a $\succeq$-maximal element of $F$ and is tax-feasible. 

I think that this holds. To be checked.

\subsubsection{Without f and g}
I now believe that the selection functions are not useful.

We assume that the lab is able to select a bundle that matches the budget and optimizes its preference. The way it proceeds may be left implicit. We want to design a tax so that the lab can pursue its normal procedure just by considering the new (taxed) costs instead of the raw costs.

\begin{theorem}
	$\forall T, M, C$,
	\begin{equation}
		\exists τ \suchthat \forall H \in T, \left[\sum_{i \in \intvl{1, k}} [(c_{a_i}) + τ(H_{\intvl{1, i}})] ≤ M ⇔ H \in F\right].
	\end{equation}
\end{theorem}
\begin{proof}
Define $τ(H_{\intvl{1, i}}) = M + 1 ⇔ \sum_{a \in H_{\intvl{1, i}}} p(a) > C$ and $τ(H_{\intvl{1, i}}) = 0$ otherwise.

We have that $H \in F ⇔ [\forall i \in \intvl{1, \card{H}}: τ(H_{\intvl{1, i}}) = 0] \land [\sum_{a \in H} c(a) ≤ M]$. 
Indeed, if $H \in F$, then $\forall i \in \intvl{1, \card{H}}: \sum_{a \in H_{\intvl{1, i}}} p(a) ≤ C$, whence the consequent; 
and if $H \notin F \land [\sum_{a \in H} c(a) ≤ M]$, then $\exists i \in \intvl{1, \card{H}} \suchthat \sum_{a \in H_{\intvl{1, i}}} p(a) > C$, whence, for such an $i$, $τ(H_{\intvl{1, i}}) = M + 1$.

To conclude, observe that $\left[[\forall i \in \intvl{1, \card{H}}: τ(H_{\intvl{1, i}}) = 0] \land [\sum_{a \in H} c(a) ≤ M]\right] ⇔ \sum_{i \in \intvl{1, k}} [(c_{a_i}) + τ(H_{\intvl{1, i}})] ≤ M$.
\end{proof}

But, that’s not the kind of tax that we want. Why? To be continued…

\begin{example}
	Consider:
	\begin{itemize}
		\item $M = 20$, $C = 2$, 
		\item $T = \set{H, H'}$, 
		\item all acts by the same member,
		\item $H = \set{a_1, a_2}$, $c(a_1) = 10$, $c(a_2) = 5$, $p(a_1) = 1$, $p(a_2) = 2$, 
		\item $H' = \set{a_1, a_3, a_4}$, $c(a_3) = 3$, $c(a_4) = 4$, $p(a_3) = 1$, $p(a_4) = 1$. 
	\end{itemize}
	All acts are raw-cost-f. Only $H$ is CO²-feasible.
	To be continued…
%	Imagine $\set{a_1, a_3}$ is just feasible and $\set{a_2, a_3}$ as well. Then need to put a tax on one of $a_1$ and $a_2$ (at least) but can’t!
\end{example}

\subsection{A simpler view on the problem}
Let us investigate a problem where we have a set $O$ of $m$ objects $o_1, \ldots, o_m$, where each object $o_i$ is associated to a triple $(u_i, c_i, p_i)$ where $u_i$ is the utility value of the item, $c_i$ is the monetary cost of the item, and $p_i$ is the pollution cost of the item. 
We make the assumption that all items are different, i.e, they correspond to different triples. 
More precisely, we investigate a $2$-dimensional knapsack instance $\mathcal{K}$ with budgets $C$ and $P$. 
Stated differently, we solve the following problem.
\begin{align}
\max_{S\subseteq O} \sum_{o_i \in S} u_i&\\
s.t.  \sum_{o_i \in S} c_i &\le C\\
 \sum_{o_i \in S} p_i &\le P\\
\end{align}

We say that $S\subset O$ satisfies Pareto-dominance if when $o_i \in S$ and $o_j \in O$ Pareto dominates $o_i$ then $o_j \in S$.
We show the following property: 
\begin{property}
$S\subset O$ satisfies Pareto-dominance iff there exists a non-decreasing function $\theta$ such that $o_i\in S$ iff $u_i \ge \theta(c_i,p_i)$.
\end{property}
\begin{proof}
$\Leftarrow$ Trivial.\\
$\Rightarrow$ We define $\theta(c,p) = \min_{u} (u,c',p')\in S$ s.t. $c'\ge c$ and $p'\ge p$. if such a couple exists and $\infty$ otherwise. The monotonicity is trivial and the fact that $o_i\in S$ iff $u_i \ge \theta(c_i,p_i)$ is due to the fact that $S$ satisfies Pareto dominance and the unicity of triples. Note that the $\theta$ function that we have defined here is the maximal possible one. A lower bound on the values of $theta$ is given by $\max_u (u,c',p')\in O\setminus S$ such that $c'\le c$ and $p'\le p$ if any exists and $0$ otherwise.  
\end{proof}

\begin{corollary}
There exists an optimal solution $S^*$ to $\mathcal{K}$ and a non-decreasing function $\theta$ such that $o_i\in S^*$ iff $u_i \ge \theta(c_i,p_i)$.
\end{corollary}

We now investigate the following problem. Given a set $S$ which satisfies Pareto dominance can we project the pollution cost on the monetary cost (with a monotone taxe) to preserve the Pareto-dominance relation. 

This condition can be checked by linear programming. 
\begin{align}
c+\tau_p \le c' + \tau_{p'} &\forall (u,c,p), (u',c',p') \in S^*\times(O\setminus S^*) ,\text{ s.t. } u \le u' \in \\
\tau_p \le \tau_p' & \forall (\_,\_,p), (\_,\_,p') \in O^2 ,\text{ s.t. }  p \le p'
\end{align} 

I will try finding a counter example. By numerical test, I know that it is not always possible but my counterexample does not consider an optimal solution to a Knapasck instance yet.

\subsection{Draft}
Here are some examples which might contribute answering some questions.
\begin{example}
	$T = \set{B_1, B_2}$ with $B_1 = B_2 \cup \set{a}$, with $a$ being top utility and low cost; then I would have to reject $a$ because it doesn’t fit the budget.
	We might have to mandate that $T$ be closed by exclusion of an element.
\end{example}

\begin{example}
	Consider acts $a$ good (in utility and CO²), $b, c$ average (in utility and CO²) and $d$ bad (in utility and CO²), all with the same bare cost, yielding the Pareto relation $a > b, a > c, b > d, c > d$.
	Consider $T = \set{\set{a, c}, \set{b, d}, \set{a}, \set{b}, \set{c}, \set{d}, \emptyset}$. 
	Consider $\set{a, c} \succ \set{b, d} \succ \set{a} \succ \set{b} \succ \set{c} \succ \set{d} \succ \emptyset$ (as determined by considering that $\succeq$ sums utilities).
	Assume that all bundles of $T$ are raw-cost-feasible.
	Consider $\set{b, d}$ over the CO² budget.
	
	We need a tax that makes $b$ and $d$ rejected but accepts $c$.
	This seems to require violating pareto-dominance.
\end{example}

Define the tax-feasible bundles as the bundles $B = (a_i)$ such that $\sum c_i + \tau_i ≤ M$.
We consider only functions $\tau$ such that the tax-feasible bundles are suitably capped.

\begin{theorem}
	Under suitable hypothesis, a flat tax is most efficient.
\end{theorem}
\begin{proof}[Intuition]
	Assume $\tau$ is not linear. To build the intuition, consider a constant tax. Consider actions $a_1$ at 1T CO², 0 € bare cost and $a_2$ at 100 kg CO², 0 € bare cost, taxed the same. Assume $a_1$ has slightly better welfare than $a_2$.
	Assume the budget is 1.1T and 1.1 k€. We need to set the constant tax to 550 €. But the welfare would be higher with $11 a_2$ (now infeasible). No, set the constant tax to 100 €. Then $a_1 + 10 a_2$ is within monetary budget and welfare optimal but out of pollution budget, so that’s rejected. There is no constant tax such that the best feasible point is $11 a_2$.
\end{proof}
		
A flat tax may spread carbon efforts in a non egalitarian way. Assume $\succeq$ has no egalitarian concerns (for example, it funds only according to cost and prestige of the venue).
Intuition. Person $P_1$ does 10 times action $a$ and person $P_2$ does 10 times action $a'$, with $a'$ an action that costs 1.1 and pollutes 1.1 as much as $a$ but that is as prestigious as $a$. These 20 actions are feasible under the given budget. Now with a flat tax that makes this bundle infeasible, the lab might want to remove as many $a'$ as necessary without touching the actions of $P_1$. This puts all carbon efforts on $P_2$.

\subsection{About quotas VS taxes}
A \emph{pollution quota} may be modeled as a function that associates to each possible act a binary decision: allowed or forbidden. Equivalently, it can be modeled as the set of allowed acts. Acts are supposedly described using the same aspects as above.

A quota function is equivalently modeled using a tax function, assuming that there is some monetary amount that nobody is able to pay, and that we write $\infty$. Such a function uses as codomain only the set $\set{0, \infty}$. Wlog, we thus only talk about tax functions in the rest of this article.

We say that a tax function is non-binary iff its co-domain is different than $\set{0, \infty}$. Note that a tax function is binary iff it corresponds to what is ordinarily called a pollution quota.

\subsection{Scope}
An act $a$ is said to be taxed iff $f(a) ≠ 0$ (in other words, the taxed acts is the support of $f$).

Some acts may be considered out of scope (e.g., impossible to tax in practice because the corresponding acts may be hidden easily by the actor, or about which a tax is illegal in some institutional context). Technically, we say that some acts are out of scope iff they are not taxed, and performing them or not never changes anything to the taxes paid about the other acts. Here is the formal definition. Given a set of acts $A$ and an act $a' \notin A$, let $a'_{-A}$ designate the act $a'$ with its context changed so as to not include the acts in $A$. Let $A_i$ designate the set of acts performed by actor $i$ within the considered time period. A set of acts $\mathcal{A}$ is out of scope iff $[\forall a \in \mathcal{A}: f(a) = 0] \land \forall a' \in A_i \setminus \mathcal{A}, \forall A \subseteq \mathcal{A}, f(a'_{-A}) = f(a')$.
An act that is not out of scope is said to be within the scope. Note that an act may be within the scope but not be taxed.

\section{LAMSADE}
The tax should use Reductionism.

The scope should include every act which the Dauphine accounting services see (to be defined more precisely?).

The tax should strictly increase with increasing act pollution (implying that no polluting act has zero tax).

Considering that efforts to not-travel are (I think) not linarly proportional to the amount of renouncement (it is easier to give up the first 10\% of ones travels than the last 10\%), it may make sense to tax exponentially (with a reasonably low exponent of course). This may permit, if suitably tuned, to make sure also the rich members of the lab make some effort (to be checked theoretically) and give some of the benefits attributed to tax quotas (at some point the tax is so high that it acts as a practical absolute interdiction). NB some members travel a lot (like Alexis used to), if they have good reasons and we want to not forbid this absolutely then it may be a problem for this proposal.

Allow for exceptions (put some further acts out of scope)? Or reduce the tax according to utility? For example, maybe the director of the lab needs to travel for some official reasons; people (or juniors only) might be encouraged to present their work at conferences; … But this may be taken into account by the increasing tax if sufficiently low at start to allow for occasional travel.

Time period? After some time (years?) the past travels should not count anymore. Or should be discounted.

Differenciate tax on one’s own ANR budget and tax on some lab-or-pole-funded act?

\subsection{Proposition}
Quoi et qui : tout membre du labo pour tout acte passant par les services comptables de Dauphine.

$\gamma(v) = c$, en T eq CO²

Acteur associé à une séquence de voyages $v_i$.

Chaque voyage $v_i$ associé à $c_i = \gamma(v_i)$ et à $t_i$, sa date.

Utilisation de l’argent de la taxe :
- financer mesures incitatives
- Payer 1ère classe
- Compenser différence Avion et moyens de transports moins polluants
- Financer des projets de recherche sur les thématiques environnementales

\subsection{Exponential tax with discounting}
Fix $λ \in (0, 1)$.

Given $α, β \in \intvl{1, n}$, define $l_α^{t_β} = \sum_{i = 0}^α c_i \lambda^{t_β - t_i}$, 
and let’s write $l_n = l_n^{t_n}$ for short.
Define $l_0 = 0$.

We have $l_n^{t_n} = \sum_{i = 0}^n c_i \lambda^{t_n - t_i} = \sum_{i = 0}^{n - 1} c_i \lambda^{t_n - t_i} + c_n = \frac{\sum_{i = 0}^{n - 1} c_i λ^{t_{n - 1} - t_i}}{λ^{t_{n - 1} - t_n}} + c_n = l_{n - 1}^{t_{n - 1}}λ^{t_n - t_{n - 1}} + c_n$.

Also, $l_{n - 1}^{t_n} = \sum_{i = 0}^{n - 1} c_i λ^{t_n - t_i} = \sum_{i = 0}^{n - 1} c_i λ^{t_{n - 1} - t_i} λ^{t_n - t_{n - 1}} = l_{n - 1}^{t_{n - 1}} λ^{t_n - t_{n - 1}}$.

Define $α = 1 + ε$ and $r = \frac{k}{\log(α)}$.

The tax of $v_1$ is defined as $\tau(v_1) = \int_0^{c_1} k α^x dx = k \frac{α^{c_1} - 1}{\log(α)} = r (α^{c_1} - 1)$.

The tax of $v_n$ is defined as $\tau(v_n) = \int_{l_{n - 1}^{t_n}}^{l_n} k α^x dx = r (α^{l_n^{t_n}} - α^{l_{n - 1}^{t_n}})$.

Considering $t_i = id$ for some constant $d > 0$ and $c_i = c$ for some constant $c$, we have $l_{n - 1}^{t_n} = c \sum_{i = 0}^{n - 1} \lambda^{t_n - t_i} = c \sum_{i = 0}^{n - 1} \lambda^{d(n - i)} = c \sum_{i = 1}^n \lambda^{di} = c (\sum_{i = 0}^n \lambda^{di} - 1) = c (\frac{1 - λ^{dn}}{1 - λ^d} - 1) = c \frac{1 - λ^{dn} - 1 + λ^d}{1 - λ^d} = c \frac{λ^d}{1 - λ^d} - c \frac{λ^{dn}}{1 - λ^d}$. 
Therefore, $τ(v_n) 
= r α^{l_{n - 1}^{t_n}} (α^{c_n} - 1) 
= r (α^c - 1) α^{c \frac{λ^d}{1 - λ^d}} α^{- c \frac{λ^{dn}}{1 - λ^d}}$.
As $λ < 1$ and $d > 0$, $λ^d < 1$, thus $\lim_{n → ∞} λ^{dn} = 0$. 
It follows that $\lim_{n → ∞} τ(v_n) = r (α^c - 1) α^{c \frac{λ^d}{1 - λ^d}}$.

We also want to adjust $r$ and $ε$ according to the length of service (ancienneté) of the actor.

\subsection{Pour fixer les paramètres de la taxe}
Half-life 2 years thus $λ^2 = 1/2$ thus $λ = \frac{\sqrt{2}}{2}$.

360 k€ (en 2020) : budget moyen projet ANR. Sur trois ans et typiquement trois institutions partenaires comprenant chacune deux chercheurs, ça fait 20 k€ / an / chercheur. Disons budget de 3 k€ / an / chercheur pour les voyages.

Prix typique d’un Paris New York semble être de 400 € A/R.

\subsection{Parameterized proposal}
Set $α = 1.4$ and $r = 847$.

An isolated trip of 0.45 T (1500 km × 2)  is then taxed 138\,€.

An isolated round-trip to Sydney (5.185\,T) is then taxed 4000\,€. This has a lasting effect: a 1.5\,T trip two years later is taxed 1330\,€, alternatively, a second round-trip to Sydney one year later is taxed 13\,735\,€.

Going to New-York (1.8\,T for the round-trip) every three months is taxed (rounding to two significant digits), for the first round-trip, 710\,€, then 1200\,€, then 2100\,€, then 3300\,€, then 5000\,€, then 7500\,€, …

Repeated 0.45\,T trips every three months are taxed from 138\,€ (first trip) to 730\,€ (approximate limit).

For a smaller tax: divide $r$ by two, then all tax amounts get divided by two.

\subsection{Loose}
\begin{itemize}
\item What and who should be taxed. Activities that are high emissions should be the ones aimed at. The level of the taxe should be dependent on how much these activities are essential to our jobs. It should be highly dependent on the stage of the career of the person being taxed.
\item What should be the level of the tax, too law is counter-productive, too high would not be accepted.  
\item How should one access the effects of the taxe and at what rate should it be changed; the taxe will probably increase in time; measuring the effects on how many planes are taken is easy.  
\item How to use tax revenues, it should be used to promote greener behaviors, it should not be green washing. 
\end{itemize}

%\bibliography{bibl}

\appendix
\section{What is done elsewhere?} 
Using the bibliographic work of Stephanie Monjon. 

\subsection{Tools to evaluate the carbon footprint}
\begin{itemize}
\item \emph{Universitetet i Bergen}: Implementation of a monitoring of emissions related to mobility with Quantified reduction targets. 
\item \emph{Stockholm Environment Institute}: Staff emissions are being monitored and can be visualized. An online tool to perform these operations is under development.
\end{itemize}

\subsection{Emissions tracking tool and reduction targets}
\begin{itemize}    
\item \emph{ETH Zurich}: Each department defined a target for the reduction of air travel (until 2025, compared to a baseline of 2016-18) and agreed on measures to achieve them. These departments Collaborate with RooteRank to facilitate the comparison of different travel options and the booking of land trips. A travel agency associated with the university is currently trained to choose more environmental options when booking trips.
\item \emph{Zurcher Hochschule der Kunste}: Each department has set its own targets and measures to reduce flight emissions by the end of 2019, exceeding the initial target of a 25\% reduction by the end of 2020 compared to a 2017 baseline.
\end{itemize}

\subsection{Offer advantages when traveling by bus or train}
\begin{itemize}
\item \emph{Universitat Autonoma of Barcelona}: For travels that take less than 10 hours, the employees are encouraged to take the train or the bus. ?These trips are presented by default to them and a travel by plane must be justified. Compensation measures are taken to increase the confort of employees choosing these modes of transport (first class, trips can be extended to add a personal trip to the professional one).
\item \emph{University of Geneve}: Subventions are granted to fund first class tickets for trips by trains that take more than 4 hours. 
\end{itemize}

\subsection{Facilitate / Recommend low-emissive modes of transport}
\begin{itemize}
\item Many universities make it possible to use decision trees to help staff members in choosing their transport mean. 
\item \emph{University of Utrecht}: A decision aiding tool was designed to help in choosing the transportation means chosen by the staff members. ?This tool takes in consideration the traveling time, the carbon footprint associated, and the number of  connections. Night trains and first class options are available. 
\item \emph{EPFL}: The administrative staff received a formation about eco-friendly traveling options. EPFL has also conceived a tool making it possible to compare different traveling options on the criteria: cost, duration, carbon footprint.
\end{itemize}

\subsection{Carbon tax to finance green projects}
\begin{itemize}
\item \emph{Arizona State University}: Since 2018, a taxe of 15\$ for any plane round trip is applied. The money which is collected is used for green compensation projects.
\item \emph{University of Neuchatel}: Since January 2019, a taxe is applied to all trips by planes. The money collected is used to fund activities related to sustainable development as well as projects reducing the carbon footprint of the university.
\item \emph{KU Leuven}: All staff members can, on a voluntary basis, pay a tax of 40 euros by tons of CO2eq for all travels by plane. This money is invested in sustainable activities.
\end{itemize}

\subsection{Quotas on CO2 emissions}
\begin{itemize}
\item \emph{LOCEAN}: Annual individual quotas for the missions organized by the laboratory ($80\%$ of the lab members voted in favor of these quotas in September 2020). These quotas are set to 10 tons of emissions per person in 2021. Quotas are not exchangeable but 4tCO2 can be passed from a year to another. These quotas are meant to decrease with the objective to reduce by half the emissions by 2030. Several exceptions exist: field studies, trips for teaching purposes, and missions of more than 30 days. PhDs and postdocs benefit from an out-of-quota trip every 2 years.
\end{itemize}

\subsection{Forbid flights when alternatives exist}
\begin{itemize}
\item \emph{LOCEAN}: Since automn 2020, it is forbidden to take flights for trips that can be made in less than 5 hours by land means. 
\item \emph{Vrije Universiteit Brussel}: It is forbidden to take flights for trips that can be made in less than 6 hours by land means. Exceptions can be made ?in exceptional circumstances. Avoiding planes is highly recommended for trips which can be made in less than 8 hours otherwise. 
\item \emph{University College London}: Domestic flights are forbidden in the geography department. 
\item \emph{HTW Berlin}: Since 2020, it is forbidden to take flights for trips that can be made in less than 6 hours by land means.
\end{itemize}

\section{Better analysis}
Replace the welfare ordering by a choice function to relax hypothesis.

\section{Loose}
a tax puts a price on carbon in order to reduce emissions, while a cap puts a limit on the quantity of emissions allowed, thus imputing a price.

Policy design considerations associated with implementing carbon taxes
include determining the tax base, which sectors to tax, where to set the tax rate, how to use tax
revenues, how to assess the impact on consumers and how to ensure that the tax achieves emissions
reduction goals.

critics say that carbon taxes are simply a way of raising government
revenue rather than providing environmental benefits, thus creating a tax rate that would not be economically efficient. For an efficient tax, the rate should be set equal to the marginal damage caused by
carbon emissions.

A carbon tax that rebates revenues can mitigate the impact on low-income households

taxes must be raised when reduction objective is not met (but can be politically difficult). So often prefer C and T.

Because cap and trade is often heralded as a market-based approach in the
political discourse, it is worth noting at the outset that both systems are equally market-based in
the sense that their effectiveness relies in affecting market behavior through emissions pricing.

Ady et al. 08 argue for tax rather than CaT (Cap and Trade). Tax is more economically efficient as it directly equalizes prices accross sources (not applicable for us). Also manages welfare better considering the uncertainty of abatment cost (didn’t get why). Considering redistribution effects: co² taxes are regressive. “Low-income households are more vulnerable to increases in the price of energy-intensive
goods, such as electricity, home heating fuels, and gasoline, because they spend a larger share of
their budget on these items compared with wealthier households.” CaT with free allowances also raise the profits and equity values, which benefits more the higher) income households.

\section{Literature}
For a review of
the advantages and disadvantages of an internationally harmonized carbon tax, many of which are
applicable to domestic carbon taxes, see Nordhaus (2007).

Carbon taxes: a review of experience and policy design considerations JENNY SUMNER, LORI BIRD \& HILLARY DOBOS Pages 922-943 | Published online: 15 Jun 2011 https://doi.org/10.3763/cpol.2010.0093 (Read by OC)

Aldy et al. (Aldy, J.E., Ley, E., Parry, I., 2008, ‘A tax-based approach to slowing global climate change’, National Tax Journal 61(3),
493 – 517, 10.17310/ntj.2008.3.09) discuss the design of carbon taxes
at the domestic level, suggesting that the efficient near-term tax is at least \$5 – 20 per tonne of CO 2 and
that the tax should be placed on upstream sources (Read by OC). (Discuss appropriate tax levels depending on discouting or supposed risks.)
 
Aldy and Pizer (Aldy, J.E., Pizer, W.A., 2009, ‘Issues in designing U.S. climate change policy’, Energy Journal 30(3), 179 – 209.) summarize the litera-
ture on the distributional impacts of carbon taxes, suggesting that recycling revenue to consumers can
reduce the distributional impact.

Metcalf et al. (2008) evaluated three methods of rebating
carbon tax revenue to consumers, and found that a lump sum per capita payment of \$274 made the
carbon tax progressive.

The ultimate design of a carbon tax is subject to the political process, and is thus unlikely to match the theoretically optimal design (see del Río and Labanderia, 2009).

A classification framework for carbon tax revenue use
Lee-Ann Steenkamp
https://doi.org/10.1080/14693062.2021.1946381 

See also: https://www.allocationclimat.fr/

\section{History}
In the early 1990s, carbon taxes emerged
in northern European countries, Finland being the first nation to adopt a carbon tax in 1990. Sub-
sequently, the Netherlands (1990), Norway (1991), Sweden (1991) and Denmark (1992) implemented
carbon taxes. After a lull of almost a decade, the UK began its Climate Change Levy (CCL) in 2001. In
recent years (before 2011), several new taxes have been introduced at the provincial or municipal levels in North
America.
The US House of Representatives
passed H.R.2454 ‘The American Clean Energy and Security Act of 2009’ (the ‘Waxman– Markey Bill’) in
June 2009, and the US Senate introduced S.1733 ‘Clean Energy Jobs and American Power Act’ (the
‘Kerry –Boxer Bill’) in September 2009. While recent federal policy has focused on carbon cap and
trade policies, Congress introduced three carbon tax bills in 2009. 1
the recently proposed French carbon tax was
designed to cover the sectors not addressed by the EU Emissions Trading Scheme (EU ETS).

\section{Petitions} 
\begin{itemize}
\item \emph{Max-Planck-Gesellschaft}: In 2018, an intern petition was launched so that the employees could pledge not to take the plane again for short distances. This petition was signed by 500 persons and was made public. 
\item \emph{Technische Universitat Berlin}: In 2019, the group Scientists4Future of TU Berlin launched a petition asking the teaching community of several universities of Berlin and Brandebourg to not take the plane for short distances. This petition obtained 1700 signatures.
\item \emph{Albert-Ludwigs-Universitat Freiburg}: An open letter was sent to the administration council of the university and signed by employees of the university. This letter was asking for a sustainable travel policy. 
\end{itemize}

\end{document}
