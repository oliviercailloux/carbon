\RequirePackage[l2tabu, orthodox]{nag}
\documentclass[french, english]{beamer}
\input{preamble/packages}
\input{preamble/redac}
\input{preamble/math_basics}
\input{preamble/math_mine}
\input{preamble/draw}
\usepackage{tcolorbox}

\setbeamertemplate{headline}[singleline]
%\setbeamertemplate{footline}[authortitle]

\title[Proposal for a tax]{An internal carbon tax for the LAMSADE: a proposal}
\subject{Carbon tax}
\keywords{economics}
\author{Olivier Cailloux \and Hugo Gilbert}
\institute[LAMSADE]{LAMSADE, Université Paris-Dauphine, PSL}
\date{\formatdate{11}{10}{2022}}

\begin{document}
\begin{frame}[plain]
	\tikz[remember picture,overlay]{
		\path (current page.south west) node[anchor=south west, inner sep=0] {
			\includegraphics[height=8mm]{Dauphine-Noir.png}
		};
		\path (current page.south east) node[anchor=south east, inner sep=0] {
			\includegraphics[height=1cm]{LAMSADE95.jpg}
		};
%		\path (current page.south) ++ (0, 4em) node[anchor=south, inner sep=0] {
%			\scriptsize\textcolor{blue}{\url{https://github.com/oliviercailloux/REPO}}
%		};
	}
	\titlepage
\end{frame}
\addtocounter{framenumber}{-1}

\section{Principle}
\begin{frame}{This presentation}
	\begin{itemize}
		\item Illustrate \emph{possible} principles and figures for a tax
		\item Not \emph{even} a result of convergence of the working group!
		\item Discussion welcome
		\item The working group needs \emph{you}!
		\item Focus on the tax itself, not on legal or implementation issues.
	\end{itemize}
\end{frame}

\frame
{
  \frametitle{An internal Carbon tax}

\textbf{Goal:} The tax should \chocolate{reduce the emissions} of Lamsade while \chocolate{promoting low-carbon traveling behaviors}. 

\

  \begin{itemize}
  \item \textbf{What is taxed?} Every \chocolate{mission} which goes through the Dauphine accounting services.
  \item \textbf{How is it defined?} Given an \chocolate{amount of emissions} in CO2eq, the tax defines an \chocolate{additional price} that should be paid to the laboratory.
  \item \textbf{Exceptions/Specificities?} Differences according to seniority or position ? E.g, junior researchers and students should have the possibility to travel more.
  \end{itemize}
}

\section{Use}
\begin{frame}{How is the collected money used?}
The money is used to \chocolate{promote low-carbon traveling behaviors}.

\

\begin{itemize} 
    \item \textbf{Confort}: Pay for \chocolate{first class} tickets when choosing long travels by \chocolate{train}. 
    \item \textbf{Compensation}: \chocolate{Compensate price difference} between plane and train/bus when they reveal more expensive.  Pay \chocolate{hotel nights} when connexions by trains require it.
    \item \textbf{Research}: Finance research initiatives (e.g., PhD position) on \chocolate{environmental research topics}.
\end{itemize}	
\end{frame}

\section{General proposition}
\begin{frame}{Definition of the tax: an example}
\textbf{Definition}:
\begin{itemize}
\item A trip $v$ at time $t$ induces \chocolate{emissions} $c$ in TCO2eq.
\item A sequence of $n$ trips induces a \chocolate{sequence} $(t_1, c_1),\ldots,(t_n,c_n)$ of \chocolate{(date, emission) pairs}.
\item Given a trip $v_n$ and a history of trips $v_1,\ldots, v_{n-1}$, \chocolate{the tax function $\tau$ returns a price in euros}.  
\end{itemize}

\

\textbf{Principles}:
\begin{itemize}
\item The more one pollutes, the more the tax increases. 
\item As time goes by the \chocolate{contribution of past trips} to the tax should \chocolate{vanish}.
\end{itemize}
\end{frame}


\begin{frame}{Definition of the tax: an example}
    \begin{itemize}
    \item Given $a,b \in [n]$, define $l_a^{b} = \sum_{i = 0}^a c_i \lambda^{t_b - t_i}$.
    \item Define $\alpha = 1 + \epsilon$ and $r = \frac{k}{\log(\alpha)}$.
    \item The tax of $v_n$ is defined as: \begin{tcolorbox}$$ \tau(v_n) = \int_{l_{n-1}^{t_n}}^{l_n^{t_n}} k \alpha^x dx = r (\alpha^{l_n^{t_n}} - \alpha^{l_{n - 1}^{t_n}}).$$\end{tcolorbox}
    \end{itemize}	
    
    \

   Comments:
   \begin{itemize}
   \item The tax increases exponentially with emissions.
   \item The contribution of past emissions vanishes exponentially with time.
   \end{itemize}

\end{frame}

\section{Concrete possibility}
\begin{frame}{A concrete example}
For senior researchers: 
    \begin{itemize}
    	\item $\lambda = \frac{\sqrt{2}}{2}$ (half-life 2 years); $\epsilon = 0.4$; $k = 142$.
	\end{itemize}
For junior researchers: 
    \begin{itemize}
    	\item $\lambda = \frac{1}{2}$ (half-life 1 year); $\epsilon = 0.4$; $k = 71$.
	\end{itemize}
\end{frame}

\begin{frame}{Some simulations}
Resulting taxes for a senior researcher.
    \begin{itemize}
	\item Going to New-York (1.8\,T for the round-trip) every three months → taxes by trip (350\,€, 620\,€, 1000\,€, 1600\,€, 2500\,€, 3700\,€, 5300\,€, …)
	\item An isolated round-trip to Sydney (5.185\,T) → tax 2000\,€. This has a lasting effect: a 1.5\,T trip two years later is taxed 670\,€, alternatively, a second round-trip to Sydney one year later is taxed 6900\,€.
	\item Repeated 0.45\,T trips (1800\,km\,×\,2) every three months → taxes from 69\,€ (first trip) to 370\,€ (approximate limit).
	\end{itemize}
\end{frame}

\begin{frame}{Some simulations}
    Starting from an empty history, I take a plane to \textbf{NY} in \textbf{January} 2023, then I go to \textbf{Toulouse} by plane in \textbf{June}, and  last I go to \textbf{Sydney} in \textbf{February} 2024. 
    \begin{itemize}
    \item Senior researcher: taxes of \textbf{352}, \textbf{88}, and \textbf{3322} euros.
    \item Junior researcher: taxes of \textbf{176}, \textbf{41}, and \textbf{1430} euros. 
    \end{itemize}
\end{frame}
\end{document}
